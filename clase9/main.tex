\documentclass[12pt, twoside]{article}
\usepackage{main}

% Aquí empieza el documento{{{
\begin{document}

\maketitle
\thispagestyle{fancy}

\noindent \textbf{FND}: Suma de conjunciones fundamentales\\
Suma de productos\\
O's de Y's\\
Están hechos de los mintérminos que son $1$ donde el FND es $1$

\noindent \textbf{FNC}: Producto de disyunciones fundamentales\\
Producto de sumas\\
Y's de O's\\
Están hechos de los maxtérminos que son $0$ donde el FNC es $0$

Esto tiene que ver con la representación de la función y no con la función en sí.
\begin{align*}
	x\oplus y &= \underbrace{x\overline{y}+\overline{x}y}_{\text{FND}}
	=\underbrace{(\overline{x}+\overline{y})(x+y)}_{\text{FNC}}
	=\sum m(1,2)=\prod M(0,3)
\end{align*}
\begin{center}
	\begin{tabular}{cc|c}
		$x$ & $y$ & $x\oplus y$\\
		\hline
		$0$ & $0$ & $0$\\
		$0$ & $1$ & $1$\\
		$1$ & $0$ & $1$\\
		$1$ & $1$ & $0$
	\end{tabular}
\end{center}
\begin{align*}
	\overline{\overline{(x\overline{y}+\overline{x}y)}} &=
	\overline{\overline{(x\overline{y})}\overline{(\overline{x}y)}}\\
	%
	\overline{\overline{(x\overline{y}+\overline{x}y)}} &=
	\overline{(\overline{x}+y)(x+\overline{y})}\\
	%
	\overline{\overline{(x\overline{y}+\overline{x}y)}} &=
	\overline{(\overline{x}x+\overline{x}\cdot\overline{y}+xy+y\overline{y})}\\
	%
	\overline{\overline{(x\overline{y}+\overline{x}y)}} &=
	\overline{(0+\overline{x}\cdot\overline{y}+xy+0)}\\
	%
	\overline{\overline{(x\overline{y}+\overline{x}y)}} &=
	\overline{\overline{x}\cdot\overline{y}+xy}\\
	%
	\overline{\overline{(x\overline{y}+\overline{x}y)}} &=
	\overline{(\overline{x}\cdot\overline{y})\overline{(xy)}}\\
	%
	\overline{\overline{(x\overline{y}+\overline{x}y)}} &=
	\boxed
	{
		(x+y)(\overline{x}+\overline{y})
	}
\end{align*}

\begin{align*}
	\overline{\overline{(x+y)(\overline{x}+\overline{y})}} &=
	\overline{\overline{(x+y)}+\overline{(\overline{x}+\overline{y})}}\\
	%
	\overline{\overline{(x+y)(\overline{x}+\overline{y})}} &=
	\overline{\overline{x}\cdot\overline{y}+xy}\\
	%
	\overline{\overline{(x+y)(\overline{x}+\overline{y})}} &=
	\overline{\overline{x}\cdot\overline{y}+x\cdot y}\\
	%
	\overline{\overline{(x+y)(\overline{x}+\overline{y})}} &=
	\overline{(\overline{x}+x)(\overline{x}+y)(\overline{y}+x)(\overline{y}y)}\\
	%
	\overline{\overline{(x+y)(\overline{x}+\overline{y})}} &=
	\overline{1(\overline{x}+y)(\overline{y}+x)1}\\
	%
	\overline{\overline{(x+y)(\overline{x}+\overline{y})}} &=
	\overline{(\overline{x}+y)(\overline{y}+x)}\\
	%
	\overline{\overline{(x+y)(\overline{x}+\overline{y})}} &=
	\overline{(\overline{x}+y)\overline{(\overline{y}+x)}}\\
	%
	\overline{\overline{(x+y)(\overline{x}+\overline{y})}} &=
	x\overline{y}+y\overline{x} = \boxed{x\overline{y}+\overline{x}y}
\end{align*}
\[f=\sum m(2,3,7)=\overline{x_1}x_2\overline{x_3}+\overline{x_1}x_2x_3+x_1x_2x_3\]
¿Quién es esta función?
\begin{center}
	\begin{tabular}{c|ccc|c|}
		& $x_1$ & $x_2$ & $x_3$ & f\\
		\hline
		$0$ & $0$ & $0$ & $0$ & $0$\\
		$1$ & $0$ & $0$ & $1$ & $0$\\
		$2$ & $0$ & $1$ & $0$ & $1$\\
		$3$ & $0$ & $1$ & $1$ & $1$\\
		$4$ & $1$ & $0$ & $0$ & $0$\\
		$5$ & $1$ & $0$ & $1$ & $0$\\
		$6$ & $1$ & $1$ & $0$ & $0$\\
		$7$ & $1$ & $1$ & $1$ & $1$\\
	\end{tabular}
\end{center}
\[=\prod M(0,1,4,5,6)=
	(\overline{x_1}+\overline{x_2}+\overline{x_3})
	(\overline{x_1}+\overline{x_2}+x_3)
	(x_1+\overline{x_2}+\overline{x_3})
	(x_1+\overline{x_2}+x_3)
	(x_1+x_2+\overline{x_3})
\]
\end{document}
%}}}
