\makeatletter
\def\input@path{{../}}
\makeatother
\documentclass[../main.tex]{subfiles}

\graphicspath{{ima/clase31}{../ima/clase31}}

% Aquí empieza el documento{{{
\begin{document}
\chapter{Ruedas y apareamientos}%

\thispagestyle{fancy}

Cantidad de grafos diferentes (No iguales) etiquetados sobre
$V=[n]$ son $2^\binom{n}{2}$ pues $|\mathbb{P}_2(V)|=\binom{n}{2}$

\begin{figure}[H]
	\centering
	\includesvg[width=0.8\linewidth]{dibujo}
\end{figure}

\[
	G=
	(
		\{a,b,c,d,e,f\},
		\{
			\{a,b.c\},
			\{b,c,f\},
			\{d,e,f\}
		\}
	)
\]

Teorema 1:

\[
	\sum_{x\in V}
	d(x)=2|E|
\]

Teorema 2:
Número de grados impar es par

Teorema 3:
Si $|V|>1$ hay al menos 2 vértices del mismo grado.

\section{Cosas de grafos}%
\label{sec:cosas_de_grafos}

Si $G$ es finito:
\begin{align*}
	\delta(G)&=\min_{x\in V}d(x)\\
	\Delta(G)&=\max_{x\in V}d(x)\\
	\\
	\text{sec.grad}:G&\longrightarrow (\mathbb{Z}^*)^{|V(G)|}\\
	\text{sec.grad}(G)&=
	(d_1,d_2,\cdots,d_{|V(G)})\\
	\text{donde }d_i&=
	\overset
	{
		\text{con $x_i\in V$}
	}
	{
		d(x_i)
	}
	\text{  y  }
	d_i \leq d_j \quad\forall i \leq j
\end{align*}
Es decir la secuencia de grados es el vector con los grados ordenados de menor
a mayor.

\definicion
Si $\delta(G)=\Delta(G)=d\Longrightarrow d(x)$ es constante y $G$ es regular
o ($d-$regular)

Ejemplo:

\begin{figure}[H]
	\centering
	\includesvg[width=0.8\linewidth]{dibujo-2}
\end{figure}

\begin{align*}
	\delta(G_1)=0 && |V(G_1)| &= 10\\
	\Delta(G_1)=3 && |E(G_1)| &= 10
\end{align*}
\[
	\text{sec.grados}(G_1)=(0,1,2,2,2,2,2,3,3,3)
\]
$G_1$ no es regular.

\begin{figure}[H]
	\centering
	\includesvg[width=0.5\linewidth]{dibujo-3}
\end{figure}

\begin{align*}
	|V(G_2)|&=6 && |E(G_2)|=12\\
	\delta(G_2)&=4=\Delta(G_2)
\end{align*}
$G_2$ es 4-regular o regular de grado 4.

\begin{figure}[H]
	\centering
	\includesvg[width=0.6\linewidth]{dibujo-4}
\end{figure}

\begin{align*}
	|V(G_3)|&= 11 \\
	|E(G_3)|&= \frac{10*3+10}{2} \\
	&= 20\\
	\\
	\delta(G_3) &= 3\\
	\Delta(G_3) &= 10
\end{align*}

Hay $2^\binom{11}{2}$ grafos etiquetados de 11 vértices con $V(G)=[11]$
$G_3$ no es regular.
\[
	\text{sec.grados}(G_3) = (3,3,3,3,3,3,3,3,3,3,10)
\]
Si $\omega_n$ de $n$.
\begin{align*}
	|V(\omega_n)|&=n+1\\
	|E(\omega_n)|&
	\overset
	{
		?
	}
	{
		=
	}
	2n
	=
	\frac{3n+n}{2}
	=\sum_{x\in V(\omega_n)} d(x)
	\\
	\delta(\omega_n)&=3\\
	\Delta(\omega_n)&=n\\
	\text{sec.grados}(\omega_n) &=
	(
		\underbrace
		{
			\overbrace
			{
				3,\cdots,3
			}^
			{
				n\text{ veces}
			}
			,n
		}_
		{
			n+1\text{ vértices}
		}
	)
\end{align*}
$\omega_3$ es una rueda regular.
$\omega_n$ no es regular si $n > 3$

\begin{figure}[H]
	\centering
	\includesvg[width=0.5\linewidth]{dibujo-5}
\end{figure}

``$\omega_2$'' no es un grafo

\begin{figure}[H]
	\centering
	\includesvg[width=0.5\linewidth]{dibujo-6}
\end{figure}

``$\omega_1$'' tampoco es un grafo

\begin{figure}[H]
	\centering
	\includesvg[width=0.5\linewidth]{dibujo-7}
\end{figure}

\begin{figure}[H]
	\centering
	\includesvg[width=0.3\linewidth]{dibujo-8}
\end{figure}

Vemos que podemos decir si
\begin{align*}
	\Delta(G) &= 0\quad \text{y}\quad |V(G)|=n\\
	G&\cong([n],\varnothing)=\phi_n\\
	\phi_0 &= (\varnothing,\varnothing)\\
	\phi_4 &= ([4],\varnothing)=
\end{align*}

\begin{figure}[H]
	\centering
	\includesvg[width=0.4\linewidth]{dibujo-9}
\end{figure}

$\phi_n$ es 0-regular
\begin{align*}
	\delta(\phi_n) &= \Delta(\phi_n)=0\\
	\text{sec.grados}(\phi_n)&=
	(
		\underbrace
		{
			0,\cdots,0
		}_
		{
			n\text{ veces}
		}
	)
\end{align*}

Si $|V(G)|=n$ y $\Delta(G)=1$

Dos casos:
\begin{itemize}
	\item $\delta(G)=0$ el grafo tiene puntos aislados.
		Si le quitamos los puntos aislados debe quedar un grafo 1-regular
		con una cantidad par de vértices de grado 1. (Como el de abajo)
	\item $\delta(G)=1$ el grafo no tiene puntos aislados.
		$G$ es 1-regular y $n$ debe ser par.
\end{itemize}

¿Cómo es un grafo $G$ finito con $\delta(G)=\Delta(G)=1$?

\begin{figure}[H]
	\centering
	\includesvg[width=0.9\linewidth]{dibujo-10}
\end{figure}

\begin{align*}
	\mathbb{A}_\frac{n}{3} &= ([n],\quad)\\
	E(\mathbb{A}_n) &=
	\{
		\{a,a+1\},
		a \notin \dot{2},
		a\in[n-1]
	\}\\
	&=
	\{
		\{a,a+1\}:
		a\notin \dot{2},
		a \in [n]
	\}
	%\{
		%\{1,2\},
		%\{3,4\},
		%\cdots,
		%\{n-1,n\}
	%\}
	%=\{
		%\{a,a+1\},
		%a \notin \dot{2},
		%a \in [n]
	%\}\\
	%|E(\mathbb{A}_n)| &= \frac{n}{2}
\end{align*}

¿Cuantos apareamientos diferentes de $\frac{n}{2}$ hay?
\[
	\overbrace
	{
		\binom{n}{ \frac{n}{2} }
	}^
	{
		\substack
		{
			\text{El lado}\\
			\text{izquierdo}
		}
	}
	\overbrace
	{
		\left(
			\frac{n}{2}
		\right)
		!
	}^
	{
		\substack
		{
			\text{Las sobras}
		}
	}
\]
$\mathbb{A}_2$:

\begin{figure}[H]
	\centering
	\includesvg[width=0.6\linewidth]{dibujo-13}
\end{figure}


\begin{figure}[H]
	\centering
	\includesvg[width=0.5\linewidth]{dibujo-11}
\end{figure}

\begin{figure}[H]
	\centering
	\includesvg[width=0.3\linewidth]{dibujo-12}
\end{figure}

\section{Tarea de grafos}%
\label{sec:tarea_de_grafos}

\begin{enumerate}
	\item Define por comprensión un conjunto de aristas para $\omega_n$
		la rueda de $n$.
	\item Determina el número de $\mathbb{A}_{\frac{n}{2}}$
		los apareamientos de $\frac{n}{2}$ para $n$ par.
\end{enumerate}

\end{document}
%}}}
