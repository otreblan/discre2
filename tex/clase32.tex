\makeatletter
\def\input@path{{../}}
\makeatother
\documentclass[../main.tex]{subfiles}

\graphicspath{{ima/clase32}{../ima/clase32}}

% Aquí empieza el documento{{{
\begin{document}
\chapter{Varios grafos}%

\thispagestyle{fancy}

\definicion
Apareamiento de $k\quad k\in \mathbb{N}\quad [2k]=\{x\in\mathbb{N}:x\leq 2k\}$
\begin{align*}
	\overset
	{
		\text{Apareamiento de $k$}
	}
	{
		\mathbb{A}_k
	}
	&=
	(
		[2k],
		\{
			\{j,j+1\}:
			j\in [2k] \wedge j \notin \dot{2}
		\}
	)
\end{align*}

\begin{figure}[H]
	\centering
	\includesvg[width=0.8\linewidth]{dibujo}
\end{figure}

\begin{figure}[H]
	\centering
	\includesvg[width=0.4\linewidth]{dibujo-2}
\end{figure}

\[
	CP_n =
	(
		[n],
		\{
			\{i,i+1\}:
			i \in[n-2]
		\}
	)
\]
Camino y punto:

\begin{figure}[H]
	\centering
	\includesvg[width=0.8\linewidth]{dibujo-3}
\end{figure}

\section{Qué pasa si}%
\label{sec:que_pasa_si}

Qué pasa si $G$ es tal que
\[\Delta(G)=1\]
\begin{itemize}
	\item Caso 1:\\
		$G$ tiene vértices aislados
		\[
			(\exists x \in V(G):d(x)=0)
		\]
	\item Caso 2:\\
		$G$ no tiene vértices aislados
		\[
			\Rightarrow n = |V(G)|\in\dot{2}
			\text{ y $G$ es 1-regular}
		\]
		\[
			G \cong \mathbb{A}_{\frac{n}{2}}
		\]
\end{itemize}

\begin{itemize}
	\item Caso 1:\\
		Sea $k$ el número de vértices aislados de $G$.
		\[
			1 \leq K \leq n-2
		\]
		\[
			n-k \text{  número de vértices de grado 1.}
		\]
		\[
			n-k \in \dot{2}
		\]
		\[
			G \cong \phi_k
			\overset
			{
				\text{Unión disjunta}
			}
			{
				\sqcup
			}
			\mathbb{A}_{\frac{n-k}{2}}
		\]
\end{itemize}

\textbf{Operaciones de grafos:}

Dado $G$ y $H$ grafos

\[
	G\cup G=
	(
	V(G)\cup V(H),
	E(G)\cup E(H)
	)
\]

\begin{figure}[H]
	\centering
	\includesvg[width=0.8\linewidth]{dibujo-4}
\end{figure}

\begin{figure}[H]
	\centering
	\includesvg[width=0.8\linewidth]{dibujo-5}
\end{figure}

Qué pasa si $G$ es tal que
\[
	n = |V(G)|\qquad\text{y}\qquad\Delta(G)=2
\]

\begin{itemize}
	\item Caso 1:
		\[
			\delta(G)=2 \Rightarrow
		\]
		$G$ 2-regular.
		\[
			\text{sec.grad}(G)=
			\{
				\overbrace
				{
					2,2,2,\cdots,2
				}^
				{
					\text{$n$ veces}
				}
			\}
		\]
		\[
			G\cong
			\overset
			{
				\substack
				{
					\text{Número de}\\
					\text{ciclos en $G$}
				}
			}
			{
				\bigsqcup_{i=1}^k
			}
			C_i \text{ donde }
			C_i \cong \mathbb{C}_{m_i} \text{ un ciclo de $m_i$}
		\]
		\[
			\sum_{i=1}^k m_i = n \text{ y } m_i \geq 3 \text { y } k \geq 1
		\]
	\item Caso 2:
		\[
			\delta(G)=1 \Rightarrow
		\]
		$G$ no tiene puntos aislados y $G$ tiene una cantidad par de vértices
		de grado 1.
		\[
			\text{sec.grad}(G)=
			\{
				\overbrace
				{
					1,1,1,\cdots,1
				}^
				{
					\substack
					{
						\text{Número par}\\
						\text{de veces $k$}
					}
				}
				,
				\overbrace
				{
					2,2,2,\cdots,2
				}^
				{
					\text{$n-k$ veces}
				}
			\}
		\]
		\[
			G \cong
			\left(
				\bigsqcup_{i=1}^\omega F_i
			\right)
			\cup
			\left(
				\bigsqcup_{j=1}^\omega H_j
			\right)
		\]
		$k$ es el número de ciclos en $G$ $\omega$ es el número de
		caminos en $G$.
		\begin{align*}
			H_j &\cong \mathbb{C}_{m_j}\\
			F_i &\cong \mathbb{P}_{n_i}\\
			\\
			\sum_{j=1}^k m_j +
			\sum_{i=1}^\omega n_j
			&= |V(G)|\\
			\\
			m_j &\geq 3 \quad n_i \geq 2
		\end{align*}
		$k$ podría ser 0. Pero $\omega \geq 1$
	\item Caso 3:
		\[
			\delta(G)=0 \Rightarrow
		\]
		$G$ tiene puntos aislados y una cantidad par de vértices de grado 1.
		\[
			\text{sec.grad}(G)=
			\{
				\overbrace
				{
					0,0,0,\cdots,0
				}^
				{
					1\leq z
				}
				,
				\overbrace
				{
					1,1,1,\cdots,1
				}^
				{
					k\in \dot{2}
				}
				,
				\overbrace
				{
					2,2,2,\cdots,2
				}^
				{
					n-k-z\geq 1
				}
			\}
		\]
\end{itemize}
\definicion
\[
	\mathbb{P}_n =
	(
		[n],
		\{
			\{i,i+1\}:
			i\in[n-1]
		\}
	)
\]

$\mathbb{P}_n$ el camino de largo $n-1$

\begin{figure}[H]
	\centering
	\includesvg[width=0.8\linewidth]{dibujo-6}
\end{figure}

\begin{align*}
	|V(\mathbb{P}_n)| &= n\\
	|E(\mathbb{P}_n)| &= n-1\\
	\\
	\delta(\mathbb{P}_n) &=
	\begin{cases}
		1 &\text{si } n \geq 2\\
		0 &\text{si } n = 1
	\end{cases}\\
	\\
	\Delta(\mathbb{P}_n) &=
	\begin{cases}
		0 &\text{si } n = 1\\
		1 &\text{si } n = 2\\
		2 &\text{si } n \geq 3
	\end{cases}\\
	\\
	\text{sec.grad}(\mathbb{P}_n) &=
	\begin{cases}
		(0) &\text{si }n=1\\
		(1,1,
		\underbrace
		{
			2,\cdots,2
		}_
		{
			n-2
		}
		) &\text{si }n \geq 2
	\end{cases}
\end{align*}

\definicion
\[
	\mathbb{C}_n = \mathbb{P}_n\cup
	(
		\{1,n\},
		\{\{1.n\}\}
	)
\]
El camino de antes pero con un vértice entre la primera y última arista.

\begin{figure}[H]
	\centering
	\includesvg[width=0.8\linewidth]{dibujo-7}
\end{figure}

\definicion
\[
	\mathbb{S}_n =
	(
		[n+1],
		\{
			\{n+1,i\}:
			i\in[n]
		\}
	)
\]
La estrella de $n$.

\begin{figure}[H]
	\centering
	\includesvg[width=0.8\linewidth]{dibujo-8}
\end{figure}

\begin{align*}
	|V(\mathbb{S}_n)| &= n+1\\
	|E(\mathbb{S}_n)| &= n\\
	\\
	\delta(\mathbb{S}_n) &=
	\begin{cases}
		0, &\text{ si }n=0\\
		1, &\text{ si }n\geq1
	\end{cases}\\
	\\
	\Delta(\mathbb{S}_n) &= n\\
	\text{sec.grad}(\mathbb{S}_n) &=
	\begin{cases}
		0, &\text{ si }n=0\\
		(
		\underbrace
		{
			1,1,\cdots,1
		}_
		{
			n
		}
		,n) &\text{ si } n > 0
	\end{cases}
\end{align*}

La rueda. Unión de una estrella y un camino cerrado.
\[
	\mathbb{W}_n \cong \mathbb{S}_n \cup \mathbb{C}_n
\]

\definicion

Dado $G=(V,E)$
\[
	\overline{G} =
	(
		V_| \mathbb{P}_{ares_2}
		\overset
		{
			\substack
			{
				\text{Todos las}\\
				\text{aristas}
			}
		}
		{
			(V)
		}
		\overset
		{
			\substack
			{
				\text{Diferencia de}\\
				\text{conjuntos}
			}
		}
		{
			\diagdown
		}
		E
	)
\]
Obsevación
\[
	G \cup \overline{G}= \mathbb{K}_V
\]

\end{document}
%}}}
