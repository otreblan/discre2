\makeatletter
\def\input@path{{../}}
\makeatother
\documentclass[../main.tex]{subfiles}

\graphicspath{{ima/clase39}{../ima/clase39}}

% Aquí empieza el documento{{{
\begin{document}
\chapter{Planaridad}%

\thispagestyle{fancy}

$G$ es planar si admite una representación plana (Si las aristas no se cruzan).

\teorema
$G$ es planar si no a tiene a $\mathbb{K}_5$ o $\mathbb{K}_{3,3}$ como menor.

\definicion
$H$ es un menor de $G$ si se puede obtener $H$ a partir de $G$ mediante
operaciones sucesivas de:
\begin{itemize}
	\item Eliminación de vértice.
	\item Eliminación de aristas.
	\item Retracción de aristas.
\end{itemize}

\begin{figure}[H]
	\centering
	\includesvg[width=\linewidth]{dibujo-1}
\end{figure}

\begin{align*}
	H &= ((V\setminus \{x,y\})\cup \{*\}
	(E\setminus\{e\in E:x\in\vee y \in e\}) \cup
	\{\{*,z\}:z\in(N_x\cup N_y)\setminus\{x,y\}\})
\end{align*}

Ejemplo:

\begin{figure}[H]
	\centering
	\includesvg[width=0.7\linewidth]{dibujo-2}
\end{figure}

$H$ es un menor de $G$. $H\cancel{\subseteq} G$

Ejemplo 2:

\begin{figure}[H]
	\centering
	\includesvg[width=0.3\linewidth]{dibujo-3}
\end{figure}

$H$ es un menor de $G$ $H \cancel{\subseteq} G$

\begin{figure}[H]
	\boldmath
	\centering
	\includesvg[width=0.3\linewidth]{dibujo-4}
\end{figure}

\[
	\mathbb{K}_5 \subseteq G
\]
$\mathbb{K}_5$ es un menor de $G$.

¿Es $\mathbb{S}_4$ un menor de $\mathbb{K}_9$?

Sí.

¿Es $\mathbb{C}_4$ un menor de $\mathbb{Q}_5$?

Sí.

Si solo se vale usar retracción de aristas.
¿Qué grafos ``terminan'' en un único vértice?

Los conexos.

Si $G$ es planar entonces
\[
	\chi(G) \leq 4
\]
Si $\neg(\chi(G) \leq 4)$ entonces $\neg(G \text{ es planar})$.

Si $\chi(G)>4$ entonces $G$ no es planar.

\textbf{(Una versión más antigua)}
Si $G$ es planar.
Entonces $\chi(G)\leq 6$

Bosquejo:

Si $G$ es planar y tiene todas las aristas que podría tener, es una
triangularización del plano.

\begin{figure}[H]
	\boldmath
	\centering
	\includesvg[width=0.6\linewidth]{dibujo-5}
\end{figure}

\teorema
\begin{align*}
	|\text{caras}| + |V| &= |E| + 2\\
	14 + 9 &= 21 +2\\
	23 &= 23
\end{align*}

\subsection*{Cosas}%
\begin{align*}
	\sum_{x\in V} d(x) &= 2|E|\\
	%\sum_{x\in V} d(x) &= 2(|\text{caras}|+|V|-2)
	3|\text{caras}| &= 2|E|
\end{align*}

$G$ planar:

$T_r$ es una triangularización del plano.
\begin{align*}
	|E(G)| \leq |E(T_r)|
\end{align*}

\section{Tarea planar}%
\label{sec:tarea_planar}

Cuando $G$ es suficientemente grande.
Si $G$ es una triangularización entonces $G$ tiene al menos un vértice con
grado 5 o más.

\begin{align*}
	\sum_{x\in V}d(x) &= 2|E|\\
	\sum_{x\in V}\delta(G) &\leq \sum_{x\in V}d(x) \leq\sum_{x\in V}\Delta(G)\\
	\sum_{x\in V}\Delta(G) &= |V|\Delta(G)\\
	\\
	2|E|=\sum_{x\in V} d(x) &\leq \sum_{x\in V}\Delta(G) = |V|\Delta(G)\\
	\frac{2|E|}{|V|} &\leq  \Delta(G)\\
	|E| &\leq \frac{(|V|-1)|V|}{2}
\end{align*}

\begin{figure}[H]
	\boldmath
	\centering
	\includesvg[width=0.6\linewidth]{dibujo-6}
\end{figure}


\end{document}
%}}}
