\makeatletter
\def\input@path{{../}}
\makeatother
\documentclass[../main.tex]{subfiles}

\graphicspath{{ima/clase35}{../ima/clase35}}

% Aquí empieza el documento{{{
\begin{document}
\chapter{Repaso de grafos}%

\thispagestyle{fancy}

\definicion

\begin{itemize}
	\item Paseo:

		Una secuencia de vértices
		\begin{gather*}
			\omega = (V_1,V_2,V_3,\cdots,V_{n-1},V_n,V_{n+1})\\
			\{V_i, V_{i+1}\} \in E(G)\quad \forall i \in[n]
		\end{gather*}
	\item Sendero:

		Un paseo que no repite aristas si es cerrado en un circuito.
	\item Camino:

		Un sendero que no repite vértices si es cerrado es un ciclo.

	\item Largo:

		Número de aristas por las que pasa.
	\item Un sendero es euleriano si visita todas las aristas de $G$
		exactamente una vez.
	\item Un camino es hamiltoniano si visita todos los vértices de $G$
		exactamamente una vez.
\end{itemize}

\definicion
$G=(V,E)$ es bipartito si existen conjuntos $A$ y $B$ tales que
$A\cap B = \varnothing \wedge A \cup B = V$

Tales que las aristas de $G$ tienen un extremo en $A$ y otro en $B$.
\begin{figure}[H]
	\centering
	\includesvg[width=0.8\linewidth]{dibujo-1}
\end{figure}
Al parecer no hay aristas entre los vértices dentro de $A$ con los de $A$.
Ni desde $B$ con los de $B$.

Todo grafo bipartito es coloreable con 2 colores o menos.

\begin{align*}
	&\mathbb{K}_{n,m}\quad\text{es bipartito}\\
	&\mathbb{A}_n\quad\text{es bipartito}\\
	&\mathbb{S}_n\quad\text{es bipartito}\\
	&\mathbb{Q}_n\quad\text{es bipartito}\\
\end{align*}

\teorema
Si $G$ tiene un ciclo impar como subgrafo entonces $G$ no es bipartito.

\begin{figure}[H]
	\centering
	\includesvg[width=0.5\linewidth]{dibujo-2}
\end{figure}

\demostracion
Supongamos que $G$ es bipartito, podemos separar
$V(G)=A\cup B, A\cap B=\varnothing$ de manera que las aristas de $G$ tienen un
extremo en $A$ y otro en $B$.
Esto induce una partición de los vértices de $\mathbb{C}_{2k+1}$
\begin{gather*}
	V(\mathbb{C}_{2k+1})=X \cup Y\\
	\text{con } X\cap Y = \varnothing\\
	X \subseteq A \wedge Y\subseteq B
\end{gather*}

Es decir las aristas de $\mathbb{C}_{2k+1}$ deben tener un extremo en $X$ y el
otro en $Y$.
Hay $2k+1$ aristas.

Digamos que $1\in X\rightarrow 2\in Y \Rightarrow 3\in X \Rightarrow \cdots
2k+1\in \mathbb{X}$

Pero $\{1,2k+1\}\in\mathbb{E}
(\mathbb{C}_{2k+1})
\subseteq E(G)\Rightarrow\Leftarrow$

\definicion
\begin{align*}
	\text{diam}(G) &= \max_{x,y} \text{ dist}(x,y)\\
	\text{dist}(G) &= \min_{xPy}
	\underbrace
	{
		| E(xPy) |
	}_
	{
		\text{Largo del camino}
	}
\end{align*}

$E(xPy)$ Conjunto de aristas del camino entre $x$ y $y$.

\begin{align*}
	\text{diam}(\mathbb{K}_n) &=
	\begin{cases}
		1&n>1\\
		0 &n=1
	\end{cases}
	\\
	\text{diam}(\mathbb{S}_n) &=
	\begin{cases}
		1 & n=1\\
		2 &n>1
	\end{cases}
	\\
	\text{diam}(\mathbb{A}_n) &=
	\begin{cases}
		\infty &n>1\\
		1 &n=1
	\end{cases}\\
	\text{diam}(\mathbb{P}_n) &= n-1\\
	\text{diam}(\omega_n) &=
	\begin{cases}
		2 &n >3\\
		1 &n =3
	\end{cases}\\
	\text{diam}(\phi_n) &=
	\begin{cases}
		\infty &n >1\\
		0 &n =1
	\end{cases}\\
	\text{diam}(\mathbb{C}_n) &=
	\begin{cases}
		\frac{n}{2} &n\in \dot{2}\\
		\lfloor \frac{n}{2} \rfloor &n \notin \dot{2}
	\end{cases}
	= \lfloor \frac{n}{2} \rfloor
\end{align*}

\end{document}
%}}}
