\makeatletter
\def\input@path{{../}}
\makeatother
\documentclass[../main.tex]{subfiles}

\graphicspath{{ima/clase28}{../ima/clase28}}

% Aquí empieza el documento{{{
\begin{document}
\chapter{Más más grafos}%

\thispagestyle{fancy}

Sea $G=(V,E)$

Un grafo:

Representación

Forma directa:

Especificamos $V$, especificamos $E$ como conjuntos.

Por ejemplo:

\begin{figure}[H]
	\centering
	\includesvg[width=0.8\linewidth]{dibujo}
\end{figure}

\begin{align*}
	V&=\{1,2,3,4,5,6,7,8\}\\
	E&=\{
		\{1,8\},
		\{2,8\},
		\{1,2\},
		\{3,4\},
		\{4,5\},
		\{4,7\}
	\}
\end{align*}
Matriz de adyacencia.

(A la izquierda de la matriz deberían estar los números del 1 al 8)
\begin{align*}
	1\quad
	2\quad
	3\quad
	4\quad
	5\quad
	6\quad
	7\quad
	8\quad\\
	\begin{bmatrix}
		0 & 1 & 0 & 0 & 0 & 0 & 0 & 1\\
		1 & 0 & 0 & 0 & 0 & 0 & 0 & 1\\
		0 & 0 & 0 & 1 & 0 & 0 & 0 & 0\\
		0 & 0 & 1 & 0 & 1 & 0 & 1 & 0\\
		0 & 0 & 0 & 1 & 0 & 0 & 0 & 0\\
		0 & 0 & 0 & 0 & 0 & 0 & 0 & 0\\
		0 & 0 & 0 & 1 & 0 & 0 & 0 & 0\\
		1 & 1 & 0 & 0 & 0 & 0 & 0 & 0
	\end{bmatrix}
\end{align*}

\begin{figure}[H]
	\centering
	\includesvg[width=0.8\linewidth]{dibujo-2}
\end{figure}

\begin{gather*}
	\text{Pero } H
	\overset
	{
		\text{Isomorfo}
	}
	{
		\cong
	}
	G\\
	``\cong'' \subseteq G' \times
	\overset
	{
		\substack
		{
			\text{El conjunto}\\
			\text{de los grafos}
		}
	}
	{
		G'
	}
\end{gather*}

Dos grafos son isomorfos si existe una biyección entre sus etiquetas que
respecta las adyacencias.

\begin{figure}[H]
	\centering
	\includesvg[width=0.4\linewidth]{dibujo-3}
\end{figure}

Respeta adyacencias significa.
\begin{gather*}
	\{x,y\}\in E(G) \Longleftrightarrow \{\varphi(x),\varphi(y)\}\in E(H)\\
	\forall x,y\in V(G)
\end{gather*}

\begin{figure}[H]
	\centering
	\includesvg[width=0.9\linewidth]{dibujo-4}
\end{figure}

\end{document}
%}}}
