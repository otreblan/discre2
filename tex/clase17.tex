\documentclass[../main.tex]{subfiles}

\graphicspath{{ima/clase17}{ima}}

% Aquí empieza el documento{{{
\begin{document}
\chapter{Mini clase}%

\thispagestyle{fancy}

\[
	\begin{cases}
		x_1+x_2+x_3+x_4=15\\
		0 < x_1 < x_2 < x_3 < x_4
	\end{cases}
\]

\definicion
Coprimos.

Su su máximo común divisor es $1$.

\section{Un problema}%
\label{sec:un_problema}

Se seleccionan $101$ enteros de $S=[200]$.
Muestra que existe al menos un número que divide a otro.

Converserse de que $x\in S$
{
	\Huge
	\[
		X=2^ny
	\]
}
con
$
	y\in
	\overbrace
	{
		\{1,3,5,7,9,\cdots,199\}
	}^
	{
		\text{$100$ números}
	}
$
(Impares)
y $n\in\{0\}\cup\mathbb{N}$

Al seleccionar $101$ números de $[200]$
$
\overbrace
{
	\text
	{
		repetimos al menos un $y$
	}
}^
{
	\text
	{
		Por hoyo-polluelo
	}
}
$

Digamos que ese $y\in y_0$.
Entonces hay dos números $x_1=2^{n_1}y_0$ y
$x_2=2^{n_2}y_0$
Son $n_1\neq n_2$ pero entonces $n_1 > n_2$ o $n_2 > n_1$
y en consecuencia
$x_2 \Big{|} x_1$ o
$x_1 \Big{|} x_2$

\begin{itemize}
	\item Hoyo: $y$
	\item Polluelos: $101$ enteros
\end{itemize}

\section{Un problema con hombres y mujeres}%
\label{sec:un_problema_con_hombres_y_mujeres}

\begin{itemize}
	\item $4$ Mujeres
	\item $4$ Hombres
\end{itemize}

Un hombre no puede ir detras de otro.
Ya que hay una mujer entre ellos.

¿De cuántas formas se pueden formar?
\[
	3*
	\overbrace
	{
		4!
	}^
	{
		\substack
		{
			\text{Permuta a}\\
			\text{las mujeres}\\
		}
	}
	\overbrace
	{
		3!
	}^
	{
		\substack
		{
			\text{Permuta a}\\
			\text{los hombres}\\
		}
	}
\]
% <++> &
\begin{center}
	\begin{tabular}{ccc}
		M & M & H\\
		H & M & M\\
		M & H & H\\
		H & M & M\\
		M & H & H\\
		H & M & M\\
		M & H & M\\
		$4!3!$ & <++> & <++>
		%<++> & <++> & <++>\\
	\end{tabular}
\end{center}



\section{Un problema con monedas}%
\label{sec:un_problema_con_monedas}

Supongan que tienen que dar el vuelto de $47$ soles.

Pueden usar monedas/billetes de:
\begin{itemize}
	\item 1 sol $x_1$
	\item 2 soles $x_2$
	\item 5 soles $x_5$
	\item 10 soles $x_{10}$
	\item 20 soles $x_{20}$
	\item 50 soles $x_{50}$
\end{itemize}

¿De cuántas formas puede hacerse esto?

\begin{gather*}
	\begin{cases}
		x_1+2x_2+5x_5+10x_{10}+20x_{20}+50x_{50} = 47\\
		0 \leq x_i \quad x_i \in \mathbb{Z}
	\end{cases}
\end{gather*}

\end{document}
%}}}
