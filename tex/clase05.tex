\documentclass[12pt, twoside]{article}

\graphicspath{{ima/clase05}{ima}}

% Aquí empieza el documento{{{
\begin{document}

\setcounter{chapter}{4}
\chapter{Órdenes parciales}%
\thispagestyle{fancy}


\begin{itemize}
	\item $A$ un conjunto.

		$R$ una relación de orden parcial.

		$(A,R)$ es un CPO (Conjunto parcialmente ordenado)

	\item Un CPO con $|A| < \infty$ tiene un diagrama de Hasse.

	\item $B \sqsubseteq A$
		$x\in A$ es una cota superior \textcolor{red}{cota inferior} para $B$.

		si $yRx \forall y\in B$
		\textcolor{red}{si $xRy \forall y\in B$}

	\item $x\in B$ es MAXIMAL \textcolor{red}{MINIMAL} si no existe $y\in B$
		tal que $xRy$ \textcolor{red}{$yRx$}

	\item $x\in B$ es un máximo \textcolor{red}{mínimo} si $x\in B$, es MAXIMAL
		\textcolor{red}{MINIMAL} y es único.

	\item R es un \underline{orden total} si $R$ es un orden parcial y
		$\forall x \vee y\ xRy \vee yRx$

	\item Dado $B \sqsubseteq A$ $X$ es la menor cota superior
		\textcolor{red}{la mayor cota inferior} de $B$ si $x$
		es el mínimo \textcolor{red}{máximo} de las cotas superiores
		\textcolor{red}{inferiores } de $B$.
\end{itemize}
\begin{enumerate}
	\item Cota superior: :v
	\item Cota inferior: :o
	\item Maximales: +
	\item Minimales: -
	\item Máximo: M
	\item Mínimo: m
	\item Menor cota superior: mcs
	\item Mayor cota superior: MCI
\end{enumerate}

\begin{figure}[H]
	\centering
	\includesvg[width=0.8\linewidth]{dibujo}
\end{figure}
\begin{itemize}
	\item Obs:\\
		La menor cota superior \textcolor{red}{mayor cota inferior}, cuando existe es única.
	\item Obs:\\
		Cuando la menor cota superior \textcolor{red}{mayor cota inferior} está
		dentro del conjunto, es el máximo \textcolor{red}{mínimo} del conjunto.
	\item Obs:\\
		Si $|B|=1$ el elemento de $B$ es la cota superior, cota inferior, maximal, minimal,
		máximo, mínimo, mcs, MCI \underline{trivialmente}.
	\item Obs:\\
		El primer \underline{caso interesante} es si $|B|=2$
	\item Dfinición:\\
		Si $\forall V \sqsubseteq A, |B|=2$ podemos definir $mcs(B)$ y $MCI(B)$
		entonces estamos en un \underline{retículo}.
\end{itemize}
\textcolor{red}{\section{¿Cuántas relaciones de orden parcial podemos establecer sobre [2]?}}
\begin{figure}[H]
	\centering
	\includesvg[width=0.5\linewidth]{dibujo2}
\end{figure}

No es retículo. No es semiretículo
\begin{figure}[H]
	\centering
	\includesvg[width=0.2\linewidth]{dibujo3}
\end{figure}

No es retículo. No es semiretículo
\begin{figure}[H]
	\centering
	\includesvg[width=0.2\linewidth]{dibujo4}
\end{figure}

Es un retículo acotado complementado distributivo
\begin{figure}[H]
	\centering
	\includesvg[width=0.2\linewidth]{dibujo5}
\end{figure}

Es un retículo acotado \textcolor{red}{no es complementario}
\textcolor{blue}{¿Es distributivo?}
\begin{figure}[H]
	\centering
	\includesvg[width=0.3\linewidth]{dibujo6}
\end{figure}

Es un retículo acotado \textcolor{red}{no es complementario}
\textcolor{blue}{¿Es distributivo?}
\begin{figure}[H]
	\centering
	\includesvg[width=0.1\linewidth]{dibujo7}
\end{figure}

No es retículo. No es semiretículo
\begin{figure}[H]
	\centering
	\includesvg[width=0.2\linewidth]{dibujo8}
\end{figure}

\begin{itemize}
	\item D.S:\\
		$(A,R)$ es un retículo definimos
		\begin{align*}
			x\vee y &= mcs(x,y)(join)\\
			x\wedge y &= MCI(x,y)(meet)
		\end{align*}
	\item Un retículo $(A,R)$ es
		\begin{itemize}
			\item \underline{\underline{DISTRIBUTIVO:}}\\
			Si
			\begin{align*}
				\forall x \forall y \forall z x \wedge (y \vee z ) &= (x \wedge y) \vee (x \wedge z)
					&& \text{y}\\
				\forall x \forall y \forall z x \vee (y \wedge z ) &= (x \vee y) \wedge (x \vee z)
			\end{align*}
		\item \underline{\underline{ACOTADO:}}\\
			Si existen $\top \in A$ y $\bot \in A$
			tales que $(\bot Rx)\wedge (xR\top)$
			para todo $x\in A$\\
			\textcolor{blue}{Hay un máximo $(\top)$ y un mínimo $(\bot )$ en $A$}
		\item \underline{\underline{COMPLEMENTADO:}}\\
			Si $\forall x \in A \exists ! y \in A$ tal que
			\begin{center}
				$x \vee y = \top$ \ y \ $x \wedge y = \bot$
			\end{center}
		\end{itemize}
\end{itemize}
\end{document}
%}}}
