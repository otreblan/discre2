\makeatletter
\def\input@path{{../}}
\makeatother
\documentclass[../main.tex]{subfiles}

\graphicspath{{ima/clase23}{../ima/clase23}}

% Aquí empieza el documento{{{
\begin{document}
\chapter{Aún más generadoras}%

\thispagestyle{fancy}

\section{Un repaso de generadoras}%
\label{sec:un_repaso_de_generadoras}

\begin{align*}
	\underset
	{
		f:\{0\}\ \cup \mathbb{N} \longrightarrow \mathbb{R}
	}
	{
		\text{Sucesión}
	}
	&\quad
	\quad
	\text{Función generadora}\\
	f_n =
	\underset
	{
		f_n=1 \quad n \geq 0
	}
	{
		(1,1,1,\cdots)
	}
	= (f_n)^\infty_{n=0}
	&\overset
	{
		\text{FGO}
	}
	{
		\longleftrightarrow
	}
	\frac{1}{1-x}\\
	f_n =
	\underset
	{
		f_n = r^n \quad n \geq 0
	}
	{
		(1,r,r^2,r^3,\cdots)
	}
	= (f_n)^\infty_{n=0}
	&\longleftrightarrow
	\frac{1}{1-rx}
\end{align*}

Si $F(x)$ y $G(x)$ son las FGO de $f$ y $g$ y $c\in\mathbb{R}$ y
$k\in\mathbb{N}\cup\{0\}$

Entonces:
\begin{gather*}
	H(x)=F(x)+G(x)\text{ es la FGO de } h=f+g\\
	\text{ es decir }
	h_n = H(x)[x^n]=
	\left(
		F(x)+G(x)
	\right)
	[x^n] = f_n+g_n
\end{gather*}

Si $K(x)=cF(x)$ es la FGO $K=cf$
\[
	K_n=K(x)[x^n]=(cF(x))[x^n]=cf_n
\]

\subsection{Derivadas e integrales}%
\label{sub:derivadas_e_integrales}

\subsubsection{Derivadas}%
\label{subsub:derivadas}


\begin{align*}
	DF(x) &= \frac{d}{dx} F(x)
	= \frac{d}{dx}
	\left(
		\sum_{i=0}^\infty f_ix^i
	\right)\\
	&
	\underset
	{
		\substack
		{
			\text{Derivada}\\
			\text{de una}\\
			\text{constante}\\
			\text{es $0$}
		}
	}
	{
		=
	}
	\sum_{i=1}^\infty \frac{d}{dx} (f_ix^i)\\
	&= \sum_{i=1}^\infty if_ix^{i-1}\\
	&
	\underset
	{
		j=i-1
	}
	{
		=
	}
	\sum_{j=0}^\infty
	(j+i)f_{j+1}x^j\\
	\frac{d}{dx} F(x) = DF(x)[x^n]
	&=(n+1)f_{n+1}\quad \forall n \geq 0
\end{align*}
\subsubsection{Integrales}%
\label{subsub:integrales}
\begin{align*}
	IF(x) &= \int F(x)dx\\
	&= \int
	\left(
		\sum_{i=0}^\infty f_ix^i
	\right)dx\\
	&= \sum_{i=0}^\infty
	\left(
		\int f_ix^idx
	\right)+c\\
	&= c+
	\sum_{i=0}^\infty
	\frac{f_i}{i+1}x^{i+1}\\
	& \underset
	{
		j=i+1
	}
	{
		=
	}
	c +
	\sum_{j=1}^\infty
	\frac{f_{j-1}}{j} x^j\\
	\int F(x)dx = IF(x)[x^n]
	&=
	\begin{cases}
		c &\text{si } n=0\\
		\frac{f_{n-1}}{n} &\text{si } n > 0
	\end{cases}
\end{align*}

\subsection{Algo con multiplicación}%
\label{sub:algo_con_multiplicacion}

\begin{gather*}
	P(x)=F(x)G(x) \text{ es la FGO }
	P_n = \sum_{j=0}^n f_j g_{n-j}
	\underset
	{
		i=n-j
	}
	{
		=
	}
	\sum_{i=0}^n f_{n-i}g_i\\
\end{gather*}
\begin{align*}
	P(x) = F(x)G(x)&=
	(
		f_0,
		f_1x^1,
		f_2x^2,
		\cdots
	)
	(
		g_0,
		g_1x^1,
		g_2x^2,
		\cdots
	)
	\\
	&= f_0g_0+
	(f_0g_1+f_1g_0)x^1
	+ (f_0g_2+f_1g_1+f_2g_0)x^2
	+\cdots\\
	&+
	(
	f_{0}g_{n},
	f_{1}g_{n-1},
	f_{2}g_{n-2},
	\cdots,
	f_{n-2}g_{2},
	f_{n-1}g_{1},
	f_{n}g_{0},
	)x^n + \cdots
\end{align*}
\[
	P(x) [x^n]
	= \sum_{j=0}^n
	f_jg_{n-j} \longleftarrow \text{ Esto se demuestra con inducción}
\]
\begin{align*}
	F(x)* \frac{1}{1-x} [x^n]
	&= \sum_{j=0}^n f_j*1\\
	&= \sum_{j=0}^n f_j\\
	&=
	\underbrace
	{
		f_0+f_1+f_2+\cdots+f_n
	}_
	{
		\substack
		{
			n+1\text{ segundos}\\
			(n+1\text{ términos})
		}
	}
\end{align*}

\subsection{Moviendo n}%
\label{sub:moviendo_n}

\[
	\left(
		x^kF(x)
	\right)
	[x^n] =
	\begin{cases}
		f_{n-k}\quad n\geq k\\
		0\quad n < k
	\end{cases}
\]

Si
\(
	\underbrace
	{
		f_n
		\text{ tiene suficientes }
	}_
	{
		k \text{ ceros}
	}
\)
ceros al pricipio.
\[
	\frac{F(x)}{x^k} [x^n]
	=f_{n+k} \quad \forall n \geq 0
\]

\begin{align*}
	(a_0,a_1,a_2,a_3) &\longleftrightarrow A(x)\\
	(
	\overbrace
	{
		0,0,\cdots,0
	}^
	{
		k \text{ ceros}
	}
	,a_0,a_1,a_2,\cdots
	)
	&\longleftrightarrow x^kA(x)\\
\end{align*}
\begin{align*}
	(b_0,b_1,b_2,\cdots) &\longrightarrow B(x) &&
	\text{y se que }b_0,b_1,\cdots,b_{k-1}=0\\
	(b_k,b_{k+1},b_{k+2},\cdots) &\longleftarrow \frac{B(x)}{x^k}
\end{align*}

\section{Ejemplos generadores}%
\label{sec:ejemplos_generadores}

\textbf{
	Queremos encontrar la FGO:
}
\begin{align*}
	S_n &=
	\sum_{i=0}^n
	\frac{i}{i+1}\\
	&= \sum_{i=0}^n \frac{i+1-1}{i+1} \\
	%
	&= \sum_{i=0}^n
	\left(
		1 + \frac{-1}{1-(-i)}
	\right)\\
	%
	&= \sum_{i=0}^n 1
	- \sum_{i=0}^n \frac{1}{1-(-i)}\\
	%
	&=
	\overbrace
	{
		n+1
	}^
	{
		\frac{1}{(1-x)^2}
	}
	-
	\underbrace
	{
		\sum_{i=0}^n
		\frac{1}{1-(-i)}
	}_
	{
		t_n
	}
	%
	\intertext{Integrando}
	\int F(x)dx &= IF(x) =
	c + \sum_{i=1}^n \frac{f_{i-1}}{i} x^i\\
	\left(
		\int F(x)dx
	\right)[x^n]
	&=
	\begin{cases}
		c &\text{si }n = 0\\
		\frac{f_{n-1}}{n} &\text{si } n > 0
	\end{cases}
	\intertext{Forma cerrada para $\sum_{i=0}^n \frac{1}{1-(-i)} $}
	\left(
		\int \frac{1}{1-x} dx
	\right)[x^n]
	&=
	\begin{cases}
		c &\text{si }n = 0\\
		\frac{1}{n} &\text{si } n > 0
	\end{cases}\\
	%
	-\ln(1-x)[x^n] &=
	\begin{cases}
		c &\text{si }n = 0\\
		\frac{1}{n} &\text{si } n > 0
	\end{cases}
	\intertext{Tomamos $c=0$}
	%
	\text{Generadoras}
	&\quad\quad
	\text{Sucesiones}\\
	-\ln(1-x) &\longrightarrow
	\left(
		c,1,
		\frac{1}{2},
		\frac{1}{3},
		\frac{1}{4},
		\cdots
	\right)\\
	%
	\frac{-\ln(1-x)}{x}
	&\longrightarrow
	\left(
		1,
		\frac{1}{2},
		\frac{1}{3},
		\frac{1}{4},
		\cdots
	\right)\\
	%
	\frac{1}{1-x}*
	\left(
		- \frac{\ln(1-x)}{x}
	\right)
	&\longrightarrow
	\left(
		1,
		1+ \frac{1}{2},
		1+ \frac{1}{2} + \frac{1}{3},
		\cdots
	\right)\\
	\\
	T(x) &= - \frac{\ln(1+x)}{x(1-x)}\\
	\Aboxed
	{
	S(x) &=
	\frac{1}{(1-x)^2}+
	\frac{\ln{(1-x)}}{x(1-x)}
	}
\end{align*}

\textbf{
	Algo sobre alguién llamado Mario.
}
\begin{align*}
	S_n &=
	\sum_{k=0}^n
	\frac{k}{k+1}\\
	&=
	\sum_{k=0}^{n-1}
	\frac{k}{k+1} + \frac{n}{n+1} \quad n \geq 1 \\
	&\begin{cases}
		S_n &= S_{n-1} +
		\underset
		{
			\substack
			{
				\text{No}
			}
		}
		{
			\frac{n}{n+1}
		}
		\quad n \geq 1\\
		S_0 &= 0
	\end{cases}\\
	\sum_{n=1}^\infty S_nx^n &=
	\sum_{n=1}^\infty S_{n-1}x^n
	+ \sum_{n=1}^\infty
	\left(
		\frac{n}{n+1} x^n
	\right)\\
	S(x) - 0 &= xS(x)
\end{align*}

\end{document}
%}}}
