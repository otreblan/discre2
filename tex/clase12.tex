\documentclass[../main.tex]{subfiles}

\graphicspath{{ima/clase12}{ima}}

% Aquí empieza el documento{{{
\begin{document}
\chapter*{Clase 11}%

\thispagestyle{fancy}

\textbf
{
	\textcolor{red}
	{
		\[
			\mathbf{ g:[n]\longrightarrow\mathbb{B} }
		\]
		\(3^n\) funciones.
		Pues para cada valor del conjunto de partida hay que decidir \(1\)
		de \underline{\underline{tres}} cosas.
		\begin{itemize}
			\item No está en el dominio.
			\item Está en el dominio y su imagen es \(0\)
		\item Está en el dominio y su imágen es \(1\)
		\end{itemize}
	}
}
\textbf
{
	\textcolor{blue}
	{
		\[
			\mathbf{f:\mathbb{R}\longrightarrow\mathbb{R}\quad f(x)}
		\]
		\(Dom_f=\mathbb{R}-\{0\}\)\\
		\(Rango_f=\mathbb{R}-\{0\}\)
	}
}
\[
	\text{Considera }k\in\mathbb{N}\quad n \in\mathbb{Z}
\]

\[
	\begin{cases}
		\sum^k_{i=1}x_i=n\\
		0 \leq x_i\\
		x_i\in \mathbb{Z}
	\end{cases}
\]
¿Cuantas soluciones tiene este problema? $E_{n,k}$
\[
	\begin{cases}
		x_1+x_2+x_3=2\\
		0 \leq x_i\\
		x_i\in \mathbb{Z}
	\end{cases}
\]
\begin{center}
	\begin{tabular}{c|c|c}
		$x_1$ & $x_2$ & $x_3$\\
		\hline
		$0$ & $0$ & $2$\\
		$0$ & $2$ & $0$\\
		$2$ & $0$ & $0$\\
		\hline
		$0$ & $1$ & $1$\\
		$1$ & $0$ & $1$\\
		$1$ & $1$ & $0$\\
		\hline
	\end{tabular}
\end{center}
\[
	\begin{cases}
		x_1+x_2+x_3+x_4=3\\
		0 \leq x_i\\
		x_i\in \mathbb{Z}
	\end{cases}
\]
\begin{center}
	\begin{tabular}{c|c|c|c}
		$x_1$ & $x_2$ & $x_3$ & $x_4$\\
		\hline
		$0$ & $0$ & $0$ & $3$\\
		$0$ & $0$ & $3$ & $0$\\
		$0$ & $3$ & $0$ & $0$\\
		$3$ & $0$ & $0$ & $0$\\
		\hline
		$0$ & $0$ & $2$ & $1$\\
		$0$ & $0$ & $1$ & $2$\\
		$0$ & $2$ & $1$ & $0$\\
		$0$ & $1$ & $2$ & $0$\\
		$2$ & $1$ & $0$ & $0$\\
		$1$ & $2$ & $0$ & $0$\\
		$2$ & $0$ & $0$ & $1$\\
		$1$ & $0$ & $0$ & $2$\\
		$2$ & $0$ & $1$ & $0$\\
		$1$ & $0$ & $2$ & $0$\\
		$0$ & $2$ & $0$ & $1$\\
		$0$ & $1$ & $0$ & $2$\\
		\hline
		$0$ & $1$ & $1$ & $1$\\
		$1$ & $0$ & $1$ & $1$\\
		$1$ & $1$ & $0$ & $1$\\
		$1$ & $1$ & $1$ & $0$\\
	\end{tabular}
\end{center}

Ana, Beto y Carlos se quieren repartir $23$ caramelos y $7$ chocolates
¿De cuántas formas pueden hacerlo?
\[
	E_{23,3}*E_{7,3}
\]
\[
	\begin{cases}
		x_{Ana}+x_{Beto}+x_{Carlos} = 23\\
		x_{Ana},x_{Beto},x_{Carlos}\in \mathbb{Z} \text{ y son no negativos}
	\end{cases}
\]

Una solución a $E_{n,k}$ se puede representar mediante un vector con $k$ coordenadas
donde las misma suman $n$ y son no negativas.

Cada vector de esa forma se puede representar usando los símbolos (Separador) y (Dulce)
como sigue primero usamos (Separador)$[k-1]$ para definir $k$ cajas.
Luego usamos $[n]$(Dulces) para representar el contenido de cada caja.

\begin{figure}[H]
	\centering
	\includesvg[width=0.8\linewidth]{dibujo-1}
\end{figure}
\[
	E_{n,k}=\binom{n+(k-1)}{k-1} = \binom{n+(k-1)}{n}
\]
Ahora algo diferente.
\[
	\begin{cases}
		\sum^k_{i=1}x_i\leq n\\
		0 \leq x_i\\
		x_i\in \mathbb{Z}
	\end{cases}
\]
\[
	\sum^n_{j=0}E_{j,k}=E_{n,k+1}
\]
\[
	\begin{cases}
		\sum^k_{i=1}x_i+\omega= n\\
		0 \leq x_i\\
		x_i\in \mathbb{Z}\\
		0 \leq \omega\\
		\omega \in \mathbb{Z}
	\end{cases}
\]
\[
	\begin{cases}
		x_1+x_2+x_3\leq 3
		0 \leq x_i\\
		x_1 \geq 0
	\end{cases}
\]
\begin{center}
	\begin{tabular}{c|c|c|c|c|c}
		$\omega$ & $x_1$ & $x_2$ & $x_3$i & $\sum x_i$\\
		\hline
		$3$ &$0$ & $0$ & $0$ & $0$ & $E_{0,3}$\\
		\hline
		\multirow{3}{*}{2} & $0$ &$0$ & $1$ & \multirow{3}{*}{1} & \multirow{3}{*}{$E_{1,3}$}\\
		&$0$ & $0$ & $1$ &\\
		&$0$ & $1$ & $0$ &\\
		&$1$ & $0$ & $0$ &\\
		\hline
		\multirow{6}{*}{1} & $0$ &$0$ & $1$ & \multirow{6}{*}{2} & \multirow{6}{*}{$E_{2,3}$}\\
		&$0$ & $0$ & $2$ &\\
		&$0$ & $2$ & $0$ &\\
		&$2$ & $0$ & $0$ &\\
		\cline{2-4}
		&$0$ & $1$ & $1$ &\\
		&$1$ & $0$ & $1$ &\\
		&$1$ & $1$ & $0$ &\\
		\hline
		\multirow{10}{*}{0} & $0$ &$0$ & $1$ & \multirow{10}{*}{3} & \multirow{10}{*}{$E_{3,3}$}\\
		&$0$ & $0$ & $3$ &\\
		&$0$ & $3$ & $0$ &\\
		&$3$ & $0$ & $0$ &\\
		\cline{2-4}
		&$0$ & $1$ & $2$ &\\
		&$0$ & $2$ & $1$ &\\
		&$1$ & $0$ & $2$ &\\
		&$1$ & $2$ & $0$ &\\
		&$2$ & $0$ & $1$ &\\
		&$2$ & $1$ & $0$ &\\
		\cline{2-4}
		&$1$ & $1$ & $1$ &\\
	\end{tabular}
\end{center}
\[
	\sum^3_{j=0}E_{j,3}=E_{3,4}
\]
Ana, Beto, Calos y Diana se quieren repartir $17$ manzanas.
Ana debe recibir al menos $3$.
Diana debe recibir al menor 5.
¿De cuántas maneras pueden repartirlas?

\[
	\begin{cases}
		x_1+x_2+x_3+x_4=17\\
		x_1\geq3\\
		x_2\geq0\\
		x_3\geq0\\
		x_4\geq5
	\end{cases}
\]
\[
	\begin{cases}
		x_1-3+3+x_2+x_3+x_4-5+5=17\\
		x_1-3\geq3-3\\
		x_2\geq0\\
		x_3\geq0\\
		x_4-5\geq5-5
	\end{cases}
\]
\[
	\begin{cases}
		\overbrace{x_1-3}^{x_1'}
		+\overbrace{x_2}^{x_2'}
		+\overbrace{x_3}^{x_3'}
		+\overbrace{x_4-5}^{x_4'}
		=17-(3+5)=9
		\\
		x_1'\geq0\\
		x_2'\geq0\\
		x_3'\geq0\\
		x_4'\geq0
	\end{cases}
\]
\[
	E_{9.4}
	=\binom{9+(4-1)}{4-1}
	=\binom{12}{3}
	= \frac{12*11*10}{3*2*1}
	= 220
\]

¿De cuántas formas diferentes podemos expresar el número $10$ como suma de $3$ naturales?

¿De cuantas formas diferentes podemos expresar el número $10$
como suma de naturales?

¿Cuantos \dobledef{tableros}{tableros válidos} de michi hay?

¿Cuántas placas de carro diferentes se pueden hacer en Perú?

\dobledef{¿Cuántos dados de $6$ caras correctos hay?}{Un dado es correcto si sus
caras opuestas siempre suman lo mismo.}

¿Cuantos dados de $12$ caras correctos hay?
\textcolor{blue}{ \textbf{Un dodecaedro sirve de dado de $12$ caras}}

¿Cuantos brazaletes de $4$ cuentas negras y $4$ cuentas blancas se pueden hacer?
\end{document}
%}}}
