\documentclass[../main.tex]{subfiles}

\graphicspath{{ima/clase18}{ima}}

% Aquí empieza el documento{{{
\begin{document}
\chapter{Aun más conteo}%

\thispagestyle{fancy}

\section{Continuación}%
\label{sec:continuacion}

De
\ref{sec:un_problema_con_monedas}

\definicion
\textbf{Funciones generadoras}

{
	\Large \textbf{IDEA:}%
}
Utilizar una ``función formal'' para llevar la cuenta de cada posibilidad,
apoyarnos en Álgebra que es algo que sabemos hacer.

\begin{center}
	\Large $11$ soles
\end{center}

\begin{table}[H]
	\centering
	\begin{tabular}{|c|c|c|c|}
		%$<++>$ & $<++>$ & $<++>$ & $<++>$\\
		S/$1$ & S/$2$ & S/$5$ & S/$10$\\
		\hline
		$11$ & $0$ & $0$ & $0$\\
		$9$ & $1$ & $0$ & $0$\\
		$7$ & $2$ & $0$ & $0$\\
		$5$ & $3$ & $0$ & $0$\\
		$3$ & $4$ & $0$ & $0$\\
		$1$ & $5$ & $0$ & $0$\\
		\hline
		$6$ & $0$ & $1$ & $0$\\
		$4$ & $1$ & $1$ & $0$\\
		$2$ & $2$ & $1$ & $0$\\
		$0$ & $3$ & $1$ & $0$\\
		\hline
		$1$ & $0$ & $2$ & $0$\\
		\hline
		$1$ & $0$ & $0$ & $1$
	\end{tabular}
	\caption{Cantidad de monedas}
\end{table}

\begin{gather*}
	\overbrace
	{
		(
		x^0
		+x^1
		+x^2
		+x^3
		+x^4
		+x^5
		+x^6
		+x^7
		+x^8
		+x^9
		+x^{10}
		+x^{11}
		)
	}^
	{
		\text{Lleva la cuenta de el $\#$ de monedas de $1$ sol}
	}
	*\\
	\overbrace
	{
		(
		x^0
		+x^2
		+x^4
		+x^6
		+x^8
		+x^{10}
		)
	}^
	{
		\text{Monto en monedas de $2$}
	}*\\
	\overbrace
	{
		(
		x^0
		+x^5
		+x^{10}
		)
	}^
	{
		\text{Monto en monedas de $5$}
	}*\\
	\overbrace
	{
		(
		x^0
		+x^{10}
		)
	}^
	{
		\text{Monto en billetes de $10$}
	}
\end{gather*}
La respuesta es el coeficiente de $x^{11}$
($\cdots+\overbrace{n}^{\text{Cantidad}}x^{\overbrace{v}^{\substack{\text{Valor de}\\
\text{la moneda}} }}+\cdots$)

Ahora con 7 soles:
\begin{table}[H]
	\centering
	\begin{tabular}{|c|c|c|c|}
		%$<++>$ & $<++>$ & $<++>$ & $<++>$\\
		S/$1$ & S/$2$ & S/$5$ & S/$10$\\
		\hline
		$7$ & $0$ & $0$ & $0$\\
		$5$ & $2$ & $0$ & $0$\\
		$3$ & $4$ & $0$ & $0$\\
		$1$ & $6$ & $0$ & $0$\\
		$2$ & $0$ & $5$ & $0$\\
		$0$ & $2$ & $5$ & $0$\\
		\hline
	\end{tabular}
	\caption{Dinero por tipo de moneda}
\end{table}

\section{Un problema con generadoras}%
\label{sec:un_problema_con_generadoras}

Se quieren repartir $11$ caramelos, $17$ chocolates y $14$ frutas.
Entre Ana, Beto, Carlos, Daniela.

\textbf{¿De cuántas formas se pueden repartir las cosas?}
\textcolor{red}
{
	\bfseries\boldmath\noindent\\
	Ana es alérgica al chocolate.\\
	Beto solo come comida en cantidades pares.\\
	Carlos debe recibir al menos $6$ caramelos y no más de $8$.\\
	Daniela quisiera al menos dos chocolates y un múltiplo de $3$ frutas.
}

Vamos a usar la regla del producto y multiplicaremos los resultados de
asignación de caramelos, chocolates y frutas.

Caramelos:
\[
	\overbrace
	{
		\left(
			\sum_{k=0}^{11}
			x^k
		\right)
	}^
	{
		\text{Ana}
	}
	\overbrace
	{
		\left(
			\sum_{k=0}^{5}
			x^{2k}
		\right)
	}^
	{
		\text{Beto}
	}
	\overbrace
	{
		\left(
			\sum_{k=6}^{8}
			x^k
		\right)
	}^
	{
		\text{Carlos}
	}
	\overbrace
	{
		\left(
			\sum_{k=0}^{11}
			x^k
		\right)
	}^
	{
		\text{Daniela}
	}
	x^{11}:27
\]

Chocolates:
\[
	\overbrace
	{
		1
	}^
	{
		\text{Ana}
	}
	\overbrace
	{
		\left(
			\sum_{k=0}^{8}
			x^{2k}
		\right)
	}^
	{
		\text{Beto}
	}
	\overbrace
	{
		\left(
			\sum_{k=0}^{11}
			x^k
		\right)
	}^
	{
		\text{Carlos}
	}
	\overbrace
	{
		\left(
			\sum_{k=2}^{11}
			x^k
		\right)
	}^
	{
		\text{Daniela}
	}
	x^{17}:72
\]

Frutas:
\[
	\overbrace
	{
		\left(
			\sum_{k=0}^{14}
			x^k
		\right)
	}^
	{
		\text{Ana}
	}
	\overbrace
	{
		\left(
			\sum_{k=0}^{7}
			x^{2k}
		\right)
	}^
	{
		\text{Beto}
	}
	\overbrace
	{
		\left(
			\sum_{k=0}^{14}
			x^k
		\right)
	}^
	{
		\text{Carlos}
	}
	\overbrace
	{
		\left(
			\sum_{k=0}^{4}
			x^{3k}
		\right)
	}^
	{
		\text{Daniela}
	}
	x^{14}:147
\]
\[
	\boxed
	{
		27*72*147
	}
\]

\end{document}
%}}}
