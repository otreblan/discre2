\makeatletter
\def\input@path{{../}}
\makeatother
\documentclass[../main.tex]{subfiles}

\graphicspath{{ima/clase21}{../ima/clase21}}

% Aquí empieza el documento{{{
\begin{document}
\chapter{Funciones generadoras}%

\thispagestyle{fancy}

Se usan para generar suceciones.(Creo)

\section{Propiedades}%
\label{sec:propiedades}

\begin{align*}
	\text{Si } A(x) &=
	\sum_{n=0}^\infty a_nx^n\\
	\text{y } B(x) &=
	\sum_{n=0}^\infty b_nx^n\\
	\text{y } c \in\mathbb{R}
\end{align*}

La función generadora de $\left(ca_n\right)^\infty_{n=0}$
es $\sum_{n=0}^\infty ca_nx^n=cA(x)=(cA)(x)$

La función generadora de $(a_n+b_n)^\infty_{n=0}$
es $\sum_{n=0}^\infty ca_nx^n=A(x)+B(x)=(A+B)(x)$

\textbf{Creo que es como una serie geométrica}
\[
	\overset
	{
		\text{Sucesión(Función)}
	}
	{
		(1,1,1,\cdots) = (1)^\infty_{n=0}
	}
	\underset
	{
		1:1
	}
	{
		\longleftrightarrow
	}
	\overset
	{
		\text{FGM}
	}
	{
		\sum^\infty_{n=0}1*x^n
	}
	= \frac{1}{1-x}
\]
\[
	(0,1,2,3,4,\cdots)
	\longleftrightarrow
	\sum_{n=0}^\infty
	nx^n
	=
	\sum_{n=1}^\infty
	nx^n
	=
	x\sum_{n=1}^\infty
	nx^{n-1}
	=
	\frac{x}{(1-x)^2}
\]
\[
	\left(
		\binom{n}{0},
		\binom{n}{1},
		\binom{n}{2},
		\cdots,
		\binom{n}{n-1},
		\binom{n}{n},
		0,
		0,
		\cdots
	\right)
	\longleftrightarrow
	(1+x)^n
	= \sum_{j=0}^n \binom{n}{j}x^ji^{n-j}
\]
\[
	(a+b)^n =
	\sum_{k=0}^n
	\binom{n}{k} a^k b^{n-k}
\]
\textbf{Sucesión geométrica multiplicada por una constante}
\[
	(c,c,c,\cdots)
	\longleftrightarrow
	\sum_{k=0}^\infty
	cx^k
	= \frac{c}{1-x}
\]
\textbf{¿Sucesión geométrica?}
\[
	(c^0,c^1,c^2,c^3,\cdots)
	\longleftrightarrow
	\sum_{k=0}^\infty
	c^kx^k
	= \frac{1}{1-cx}
\]
\textcolor{red}
{
	\bfseries
	Idea: Usar Taylor para más series.
}
\[
	\sum_{n=0}^\infty
	\frac{x^n}{n!}
	= e^x
\]
\textcolor{blue}
{
	\bfseries
	Serie de Maclaurin: (Taylor centrada en uno)
}
\[
	f(x) =
	\sum_{k=0}^\infty
	\frac{f^{(k)}(0)x^k}{k!}
\]
\textbf{Fracciones parciales}
\[
	F(x)=
	\frac{1}{(1-x)}
	\frac{2}{(1+x)^2}
	=
	\frac{A}{1-x}
	+
	\frac{B}{1+x}
	+
	\frac{C}{(1+x)^2}
\]
\[
	\frac{A(1+x)^2+B(1-x)(1+x)+C(1-x)}{(1-x)(1+x)^2}
	=
	\frac{2}{(1-x)(1+x)^2}
\]
\[
	A(1+2x+x^2)+B(1-x^2)+C(1-x)
	=
	2+0*x+0*x^2
\]
Sistema de ecuaciones:
\begin{align*}
	\begin{cases}
		A+B+C &= 2\\
		2A-C &= 0\\
		A-B &= 0
	\end{cases}
	\Rightarrow
	\substack
	{
		C = 2A\\
		A = B
	}
	\Rightarrow
	\substack
	{
		A = \frac{1}{2}\\
		B = \frac{1}{2} \\
		C = 1
	}
\end{align*}
\[
	F(x) = \frac{1}{2}
	* \frac{1}{1-x}
	+
	\frac{1}{2}
	*
	\frac{1}{1+x}
	+
	1
	*
	\frac{1}{(1+x)^2}
\]
Coeficiente en $x^n$
\begin{align*}
	F(x)[x^n]
	&= \frac{1}{2}
	+
	\frac{1}{2}
	*(-1)^n
	+
	\frac{1}{(1+x)^2}
	[x^n]\\
	\frac{1}{2}
	+
	\frac{1}{2}
	(-1)^n
	-(n+1)(-1)^{n+1}
\end{align*}

Si $F(x)$ es la función generadora para $(f_0)_{n=0}^\infty$

$F'(x)$ es la función generadora para $\left((n+1)f_{n+1}\right)_{n=0}^\infty$
\[
	F'(x)=
	\frac{d}{dx}
	\left(
		\sum_{k=0}^\infty
		f_kx^k
	\right)
	=
	\sum_{k=1}^\infty
	\frac{d}{dx}
	(f_kx^k)
	=
	\sum_{k=1}^\infty
	kf_kx^{k-1}
	\overset
	{
		( n=k-1 )
	}
	{
		=
	}
	\sum_{n=0}^\infty
	(n+1)
	f_{n+1}x^n
\]
\end{document}
%}}}
