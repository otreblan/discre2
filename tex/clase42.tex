\makeatletter
\def\input@path{{../}}
\makeatother
\documentclass[../main.tex]{subfiles}

\graphicspath{{ima/clase42}{../ima/clase42}}

% Aquí empieza el documento{{{
\begin{document}
\chapter{Árboles requiem}%

\thispagestyle{fancy}

\begin{itemize}
	\item Árbol: $T$ un grafo acíclico y conexo.
	\item Árbol enraizado:
		\(
			(
			\underset
			{
				\text{Raiz}
			}
			{
				x
			}
			,T)
		\)
		con $x\in V(T)$
	\item Árbol $k$-ario: Un árbol donde todos los vértices que tienen
		descendientes tienen $k$ descendientes.
	\item Recorrido en pre-orden:
		\begin{itemize}
			\item $1^{\text{ro}}$ la raiz
			\item $2^{\text{do}}$ después los sub-árboles de izquierda a
				derecha en pre-orden.
		\end{itemize}
	\item Recorrido en post-orden:
		\begin{itemize}
			\item $1^{\text{ro}}$ los subárboles de izquierda a derecha
				en post-orden.
			\item $2^{\text{do}}$ último la raíz.
		\end{itemize}
\end{itemize}

\section{Si el árbol es binario}%
\label{sec:si_el_arbol_es_binario}
(2 hijos o ninguno)

\subsection{Recorrido en orden}%
\label{sub:recorrido_en_orden}

\begin{itemize}
	\item $1^{\text{ro}}$ \nameref{sub:recorrido_en_orden} del sub-árbol
		izquierdo.
	\item $2^{\text{do}}$ la raíz.
	\item $3^{\text{ro}}$ \nameref{sub:recorrido_en_orden} del sub-árbol derecho.
\end{itemize}

\[
	1,2,14,15,16,17,18,13,3,12,4,5,10,11,7,6,8,19,9
\]

\begin{figure}[H]
	\centering
	\includesvg[width=0.8\linewidth]{dibujo-1}
\end{figure}

Dado un recorrido en orden de un árbol binario.
¿Podemos reconstruir el árbol binario?

\[
	(((a+b)*c)+(d*e)*((g+h)+(e+f)))
\]

\begin{figure}[H]
	\bfseries
	\boldmath
	\centering
	\includesvg[width=0.8\linewidth]{dibujo-2}
	\caption{Árbol 1}%
	\label{fig:arbol_1}
\end{figure}

\[
	\left(
		\left(
			\left(
				\left(
					\left(
						a +
						\left(
							b*c
						\right)
					\right)
					+
					\left(
						\left(
							d*e
						\right)
						*g
					\right)
				\right)
				+ h
			\right)
			+ e
		\right)
		+ f
	\right)
\]

\begin{figure}[H]
	\bfseries
	\boldmath
	\centering
	\includesvg[width=0.8\linewidth]{dibujo-3}
	\caption{Árbol 2}%
	\label{fig:arbol_2}
\end{figure}


\section{Notación polaca inversa}%
\label{sec:notacion_polaca_inversa}

Patrón del \nameref{fig:arbol_1}:
\[
	ab+c*de*+gh+ef++*
\]

¿Cuántos patrones diferentes podemos hacer con los símbolos $\{a,b,c,*,+\}$
usándolos exactamente una vez?

¿A cuantas expresiones matemáticamente distintas corresponden?

\section{Cosas con árboles}%
\label{sec:cosas_con_arboles}

Sea $G$ un grafo conexo:
\teorema
\dobledef
{%
	Existe $T$ un árbol que es subgrafo de $G$ e incluye todos sus vértices.%
}
{%
	Un árbol de expansión.%
}

\begin{figure}[H]
	\centering
	\includesvg[width=0.8\linewidth]{dibujo-4}
\end{figure}

\demostracion
Quita una arista sobre un ciclo hasta que no haya ciclos.

\begin{figure}[H]
	\centering
	\includesvg[width=0.8\linewidth]{dibujo-5}
\end{figure}

\subsection{BFS}%
\label{sub:bfs}
Búsqueda en amplitud o por anchura.

\url{https://es.wikipedia.org/wiki/Búsqueda_en_anchura}

\subsection{DFS}%
\label{sub:dfs}
Búsqueda en profundidad.

\url{https://es.wikipedia.org/wiki/Búsqueda_en_profundidad}

¿Son el árbol \nameref{sub:bfs} y \nameref{sub:dfs} únicos?

\subsection{Dijkstra}%
\label{sub:dijkstra}

\url{https://es.wikipedia.org/wiki/Algoritmo_de_Dijkstra}

\begin{figure}[H]
	\centering
	\includesvg[width=0.8\linewidth]{dibujo-6}
\end{figure}


\end{document}
%}}}
