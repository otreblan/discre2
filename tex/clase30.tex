\makeatletter
\def\input@path{{../}}
\makeatother
\documentclass[../main.tex]{subfiles}

\graphicspath{{ima/clase30}{../ima/clase30}}

% Aquí empieza el documento{{{
\begin{document}
\chapter{Grafos y sub-grafos}%

\thispagestyle{fancy}

\begin{figure}[H]
	\centering
	\includesvg[width=0.8\linewidth]{dibujo}
\end{figure}

Demuestra que $H$ y $G$ son isomórfos o di las razones por las que no lo son.

\section{Grado}%
\label{sec:grado}

\begin{align*}
	G&=(V,E)\\
	\\
	d_G(x)&=
	\left|
		V_x
	\right|\\
	\\
	\overset
	{
		\substack
		{
			\text{Vecindad de}\\
			\text{vértices de $x$}
		}
	}
	{
		\underset
		{
			\substack
			{
				\text{Los vértices que}\\
				\text{están conectados}\\
				\text{directamente a $x$}
			}
		}
		{
			V_x
		}
	}
	&= \{y:\{x,y\}\in E\}\\
	\\
	\overset
	{
		\substack
		{
			\text{Vecindad de}\\
			\text{aristas de $x$}
		}
	}
	{
		\underset
		{
			\substack
			{
				\text{Las aristas que}\\
				\text{inciden sobre $x$}
			}
		}
		{
			E_x
		}
	}
	&= \{e\in E:enx\neq\varnothing\}
\end{align*}
Si $G$ es un grado simple, sin bucles y no-dirigido.
\[
	|V_x| = |E_x|
\]

En un grafo con bucles, cada bucle aporta 2 al grado del vértice donde está.

\begin{figure}[H]
	\centering
	\includesvg[width=0.8\linewidth]{dibujo-2}
\end{figure}
\section{Teoremas de grafos}%
\label{sec:teoremas_de_grafos}

En un grafo $G=(V,E)$ finito:
\begin{enumerate}
	\item\label{item:grados}
		\[
			\overbrace
			{
				\sum_{x\in V} d(x)
			}^
			{
				\substack
				{
					\text{Sumatoria de todos}\\
					\text{los grados}
				}
			}
			=
			\overbrace
			{
				2|E|
			}^
			{
				\substack
				{
					\text{Dos veces el}\\
					\text{número de aristas.}
				}
			}
		\]
		\subitem \textbf{Demostración:}
			Cuando sumamos cada grado estamos contando cada arista 2 veces,
			una por cada vértice que la toca.
		\item\label{item:vert_par}
		El número de vértices de grado impar es par.
	\item En $G=(V,E)$ un grafo finito.
		Tal que $|V| > 1$ existen al menos 2 vértices del mismo grado.
		\subitem \textbf{Demostración:}
			Supongamos que $G$ no tiene vértices aislados.\\
			Claramente $1\leq d(x) \leq |V| - 1$\\
			Por hoyo-polluelo, como hay
			\(
				\overbrace
				{
					|V|-1 \text{ grados}
				}^
				{
					\text{Hoyos}
				}
			\)
			diferentes y
			\(
				\overbrace
				{
					|V| \text{ vértices}
				}^
				{
					\text{Polluelos}
				}
			\)
			,al menos 2 vértices comparten el mismo grado.\\
			Si $G$ tiene vértices aislados $0\leq d(x)\leq |V|-2$.\\
			Y por hoyo-polluelo de nuevo al menos 2 vértices comparten grados.
\end{enumerate}
Demostración del teorema \ref{item:grados}:

\begin{figure}[H]
	\centering
	\includesvg[width=0.8\linewidth]{dibujo-3}
\end{figure}

Demostración del teorema \ref{item:vert_par}:

Sabemos por T\ref{item:grados} que:
\[
	\sum_{x\in V} d(x) = 2|E|
\]
Podemos separar la suma $\sum_{x\in V}d(x)$ en 2 términos.
\[
	\overbrace
	{
		\underset
		{
			d(x)\in\dot{2}
		}
		{
			\sum_{x\in V}
		}d(x)
	}^
	{
		\substack
		{
			\text{Sumatoria de los}\\
			\text{grados pares.}
		}
	}
	+
	\overbrace
	{
		\underset
		{
			d(x)\notin\dot{2}
		}
		{
			\sum_{x\in V}
		}d(x)
	}^
	{
		\substack
		{
			\text{Sumatoria de los}\\
			\text{grados impares.}
		}
	}
	=2|E|
\]

Supongamos que hay una cantidad impar de vértices de grado impar.
Eso significa que
\(
	\underset
	{
		d(x)\notin\dot{2}
	}
	{
		\sum_{x\in V}
	}d(x)
\)
es impar.

Como
\(
	\underset
	{
		d(x)\in\dot{2}
	}
	{
		\sum_{x\in V}
	}d(x)
\)
es par eso implica que
\(
	\underset
	{
		d(x)\notin\dot{2}
	}
	{
		\sum_{x\in V}
	}d(x)
	+
	\underset
	{
		d(x)\in\dot{2}
	}
	{
		\sum_{x\in V}
	}d(x)
\)
es impar y esto es una contradicción pues sabemos que la suma es $2|E|$,
un número par.

\section{Conectitud y sub-grafos}%
\label{sec:conectitud_y_sub_grafos}

\definicion
$G=(V,E)$ un grafo finito.
\begin{align*}
	\Delta(G)&\coloneqq\max_{x\in V}(d(x)) && \text{Grado máximo}\\
	\delta(G)&\coloneqq \min_{x\in V}(d(x)) && \text{Grado mínimo}
\end{align*}
Observación:(teoremas fáciles)

\[
	0\leq \delta(G) \leq d(x) \leq \Delta(G) \leq |V| - 1
\]

Si $\delta(G)=\Delta(G)=d\Longrightarrow d(x)=d$ es constante
y diremos que $G$ es un grado $d-$REGULAR.

\definicion
Si $G$ es 0-REGULAR.
Entonces $G\cong \phi_n$ para algún $n\in \mathbb{N}^*$.

Donde
\(
	\underbrace
	{
		\phi_n=([n].\varnothing)
	}_
	{
		n\text{ vértices aislados}
	}
\)
para $n\geq 1$,
o
\(
	\phi =
	\underbrace
	{
		(\varnothing,\varnothing)
	}_
	{
		\text{El grafo vacío}
	}
\)

\begin{figure}[H]
	\centering
	\includesvg[width=0.8\linewidth]{dibujo-4}
\end{figure}

\end{document}
%}}}
