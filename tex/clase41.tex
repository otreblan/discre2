\makeatletter
\def\input@path{{../}}
\makeatother
\documentclass[../main.tex]{subfiles}

\graphicspath{{ima/clase41}{../ima/clase41}}

% Aquí empieza el documento{{{
\begin{document}
\chapter{Más árboles}%

\thispagestyle{fancy}

\section{Repaso árboles}%
\label{sec:repaso_arboles}

Un árbol es un grafo conexo y acíclico.

\teorema
Si $T$ es un árbol $|E(T)|=|V(T)|-1$

\teorema
Si $T$ es un árbol entonces:
\begin{itemize}
	\item $T$ es minimalmente conexo.
	\item $T$ es maximalmente acíclico.
	\item $T$ es bipartito.
	\item $T$ es planar.
\end{itemize}

\teorema
Si $T$ es un árbol existe un único camino entre $x$ y $y$
para todo $x,y \in V(T)$

\teorema
Si $G=(V,E)$ es planar entonces $|C|+|V|=|E|+2$

\teorema
Si $T$ es un árbol y $x\in V(T)$ con $d(x)=1$ diremos que $x$ es una hoja.

\teorema
Si $G$ es planar, podemos definir.
El $\text{dual}(G)$ es planar.
Y se define.
\dobledef
{
	Los vértices del $\text{dual}(G)$ son las caras de $G$ y hay una ``arista''
	entre dos caras si comparten una aristas en $G$.
}
{
	Cuidado, podría no ser un grafo.
}

\begin{figure}[H]
	\centering
	\includesvg[width=0.6\linewidth]{dibujo-1}
\end{figure}

\begin{figure}[H]
	\centering
	\includesvg[width=0.6\linewidth]{dibujo-2}
\end{figure}

\begin{figure}[H]
	\centering
	\includesvg[width=0.6\linewidth]{dibujo-3}
\end{figure}

\begin{figure}[H]
	\bfseries
	\boldmath
	\centering
	\includesvg[width=0.8\linewidth]{dibujo-4}
\end{figure}

\begin{figure}[H]
	\bfseries
	\boldmath
	\centering
	\includesvg[width=0.8\linewidth]{dibujo-5}
\end{figure}

\conjetura

¿El $\text{dual$(\text{dual$(G)$})=G$}$?

\[
	\text{NO}
\]

¿Cuando es $\text{dual$(\text{dual$(G)$})=G$}$?

Cuando las caras de la cosa isomorfa a  $\text{dual$(G)$}$ son las mismas.

\textbf{Recorrido de un árbol.}

Un árbol con raíz o enraizado es un par $(x,T)$ con $x\in V(T)$

\ejemplo

\begin{figure}[H]
	\centering
	\includesvg[width=0.8\linewidth]{dibujo-6}
\end{figure}

Hay otras definiciones asociadas a árboles.

Árbol dirigido:

\begin{figure}[H]
	\centering
	\includesvg[width=0.8\linewidth]{dibujo-7}
\end{figure}

\definicion
Un bosque es un grafo acíclico.

\teorema
Un bosque es un ``conjunto de árboles''.
\[
	G=\bigsqcup_{i=1}^{\text{nc}(G)} T_i \quad \text{con $T_i$ un árbol.}
\]

Si dibujamos un árbol enraizado con la raíz arriba y los vértices organizados
en niveles y aristas solo entre niveles consecutivos.

(¿Se puede siempre?)
(¿De cuántas maneras podemos pintarlo?)

El número de niveles es la altura del árbol.

(¿Sería más sensato pensar en el número de niveles menos 1?)

\begin{figure}[H]
	\centering
	\includesvg[width=\linewidth]{dibujo-8}
\end{figure}

Si tenemos un árbol enraizado pintado de la manera que describrimos en la
imágen.

Hay varias formas de recorrerlo:
\begin{itemize}
	\item Pre orden:
		\subitem Primero la raíz y entonces sus sub-árboles de izquierda
			a derecha en pre orden.
			\[
				T:\quad 6,3,8,2,1,5,4,9,7
			\]
	\item Post orden:
		\subitem De izquierda a derecha los sub-árboles en post orden y
			finalmente la raíz.
			\[
				T:\quad 8,1,2,3,4,9,7,5,6
			\]
\end{itemize}

\definicion
Un árbol enraizado es $k$-ario si cada vértice tiene exactamente $k$ hijos o
ninguno.

\begin{figure}[H]
	\centering
	\includesvg[width=\linewidth]{dibujo-9}
\end{figure}

Si $T$ es $k$-ario:

Cuando la diferencia entre el nivel mínimo y el nivel máximo de dos hojas
es menor o igual que 1 diremos que el árbol es balanceado.

¿Tiene sentido pensar en esa idea de ``balanceo'' en árboles generales?

¿Cuál es la altura de un árbol $k$-ario balanceado?

\section{Más tareas de árboles}%
\label{sec:mas_tareas_de_arboles}

¿Qué es una cara en un grafo planar?

\end{document}
%}}}
