\makeatletter
\def\input@path{{../}}
\makeatother
\documentclass[../main.tex]{subfiles}

\graphicspath{{ima/clase36}{../ima/clase36}}

% Aquí empieza el documento{{{
\begin{document}
\chapter{Coloración}%

\thispagestyle{fancy}

Un grafo $G$ es bipartito si existen conjuntos $A$ y $B$ tales que
$V(G)=A\cup B \quad A\cap B=\varnothing$ y todas las aristas de $G$ tienen un
extremo en $A$ y $B$.

\begin{align*}
	\text{dist}_G (x,y) &= \min_{xPy}|E(xPy)|\\
	\text{diam} (x,y) &= \max_{x,y}|dist(x,y)|
\end{align*}

\section{Coloraciones}%
\label{sec:coloraciones}

\definicion
Una $m$-coloración(propia) de los vértices de un grafo es una función
$f:V(G)\longrightarrow[m]$ tal que los extremos de las aristas de $G$ son de
colores diferentes.

Es decir si $\{x,y\}\in E(G)\quad f(x) \neq f(y) \quad\forall \{x,y\}\in E(G)$

\begin{figure}[H]
	\centering
	\boldmath
	\includesvg[width=0.6\linewidth]{dibujo-1}
\end{figure}

\teorema
Si $G$ es bipartito entonces $G$ no tiene ciclos impares.

\teorema
Si $G$ es bipartito entonces existe una 2-coloración de los vértide de $G$.

\teorema
Si $G$ admite una 2-coloración entonces $G$ es bipartito.

$\mathbb{K}_n$ admite una $n$-coloración.

$\mathbb{K}_n$ no admite ninguna $m$-coloración con $m<n$

\definicion

$\chi(G)$: número cromático de $G$.

Es el mínimo número de colores con los que se puede hallar una coloración
propia de $G$.

$G$ debe admitir una $\chi(G)$-coloración y no debe admitir coloraciónes con
menos colores.

\begin{align*}
	\chi(\mathbb{Q}_3) &= 2
\end{align*}

\begin{figure}[H]
	\centering
	\boldmath
	\includesvg[width=0.6\linewidth]{dibujo-2}
\end{figure}

\teorema
\[
	\chi(G) > 1 \text{ si }|E(G)|\geq 1
\]
Colorario:
\[
	\chi(\phi_n)=1
\]

\begin{align*}
	|E(G)|\geq 1 &\Rightarrow \chi(G)>1\\
	\neg(\chi(G)>1) &\Rightarrow \neg(|E(G)|\geq 1)\\
	\chi(G) \leq 1 &\Rightarrow |E(G)|<1\\
	\\
	\chi(G)=1 &\qquad |E(G)|=0
\end{align*}

\definicion
\dobledef
{
	Un grafo es $k$-partito si existe una $k$-coloración de sus vértices.%
}
{%
	Sin embargo: $|V(G)|\geq k$%
}

De manera equivalente deben existir. $k$ conjuntos $A_i$ con $i\in[k]$ tales
que:
\[
	\bigcup_{i=1}^k = V(G) \text{ y } A_i\cap A_j=\varnothing\quad\forall i\neq j
\]
y las aristas de $G$ tienen extremos en conjuntos distintos.

\begin{figure}[H]
	\centering
	\boldmath
	\includesvg[width=0.6\linewidth]{dibujo-3}
\end{figure}

\teorema
\dobledef{$\text{cono}(G)$ solo es bipartito si $G=\phi_n$}{Para crear un cono
hay que tomar un grafo $G$, agregarle un vértice y añadir vértices entre
todos los vértices de $G$ y el nuevo vértice.}

\conjetura
\[
	\chi(\text{cono}(G))=\chi(G)+1
\]

\subsection*{Conjetura}%
\label{sub:conjetura}

\[
	\chi(\text{cono}(G))\leq\chi(G)+1
\]

\demostracion
De manera trivial coloreamos $G$ con $\chi(G)$ colores y el vértice adicional
con el color nuevo.

\[
	\substack
	{%
		\text{¿Podría}\\
		\text{esto ser}\\
		\text{verdad?}
	}
	\quad
	\chi(\text{cono}(G)) < \chi(G)
\]
No. Por lo tanto
\[
	\chi(\text{cono}(G)) \geq \chi(G)
\]

¿Será que podemos decir $\chi(\text{cono}(G))=\chi(G)+1$ siempre?

\begin{enumerate}
	\item $\chi(\text{cono}(G))\leq \chi(G)+1$
	\item $\chi(\text{cono}(G))\geq \chi(G)$
	\item Hay grafos donde $\chi(\text{cono}(G))\neq\chi(G)$
\end{enumerate}

Sea $f$ una coloración de $\text{cono}(G)$ con el mínimo número posible de
colores (son $\chi(\text{cono}(G))$ colores)

¿De qué color es el
\(
	\underset
	{
		x_0
	}
	{
		\text{vértice}
	}
\)
de $\text{cono}(G)$?

El vértice de $\text{cono}(G)$ es de un color diferente a todos los colores de
los vértices con la coloración propia que da $f$.

Esto quiere decir que $f(x)\neq f(x_0)\quad\forall x\in V(G)$ por lo que $f$
colorea a $G$ con $\chi(\text{cono}(G))-1$.

\begin{align*}
	\chi(\text{cono}(G)) - 1 &\geq \chi(G)\\
	\chi(\text{cono}(G)) &\geq \chi(G)+1\\
\end{align*}

Usando \nameref{sub:conjetura}:
\[
	\chi(\text{cono}(G)) = \chi(G)+1
\]

\begin{align*}
	V(G_n) &=
	\left\{
		\underbrace
		{
			\overbrace
			{
				x_{n-1}x_{n-2}\cdots x_2 x_1 x_0
			}^
			{
				\substack
				{
					\text{Secuencias en}\\
					\text{ternario}\\
					\text{con $n$ trits.}\\
				}
			}
		}_
		{
			\text{$n$-trits}
		}
		:
		\underset
		{
			i\in[n-1]\cup\{0\}
		}
		{
			x_i \in\{0,1,2\}
		}
		\right\}\\
		E(G_n) &=
		\left\{
			\{\vec{x},\vec{y}\}:
			\substack
			{
				\text{Si $\vec{x}$ y $\vec{y}$}\\
				\text{difieren}\\
				\text{exactamente}\\
				\text{en unn trit.}\\
			}
		\right\}
\end{align*}


\begin{figure}[H]
	\centering
	\boldmath
	\includesvg[width=0.4\linewidth]{dibujo-4}
\end{figure}

\begin{figure}[H]
	\centering
	\boldmath
	\includesvg[width=0.9\linewidth]{dibujo-5}
\end{figure}


\end{document}
%}}}
