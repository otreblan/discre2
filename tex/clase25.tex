\makeatletter
\def\input@path{{../}}
\makeatother
\documentclass[../main.tex]{subfiles}

\graphicspath{{ima/clase25}{../ima/clase25}}

% Aquí empieza el documento{{{
\begin{document}
\chapter{Mini clase 2}%

\thispagestyle{fancy}

\section{Algo con generadoras exponenciales}%
\label{sec:algo_con_generadoras_exponenciales}

{
	\centering
	\Huge
	ANTIDERIVADA

}
\begin{itemize}
	\item A: 3
	\item N: 1
	\item T: 1
	\item I: 2
	\item D: 2
	\item E: 1
	\item R: 1
	\item V: 1
\end{itemize}

Resolver simultáneamente\\
\# de formas de permutar las letras de cualquier sub-palabra de
ANTIDERIVADA.
De largo $K$
\begin{align*}
	\intertext{$K=12$}
	\frac{12!}{3!(2!)^2(1!)^5} &=
	\binom{12}{3,2,2,1,1,1,1,1}\\
	&=
	\binom{12}{3}
	\binom{9}{2}
	\binom{7}{2}
	\binom{5}{1}
	\binom{4}{1}
	\binom{3}{1}
	\binom{2}{1}
	\binom{1}{1}\\
	&=
	\frac{12!}{3!\cancel{9!}}
	*\frac{\cancel{9!}}{2!\cancel{7!}}
	*\frac{\cancel{7!}}{2!\cancel{5!}}
	*\frac{\cancel{5!}}{1!\cancel{4!}}
	*\frac{\cancel{4!}}{1!\cancel{3!}}
	*\frac{\cancel{3!}}{1!\cancel{2!}}
	*\frac{\cancel{2!}}{1!\cancel{1!}}
	*\frac{\cancel{1!}}{1!0!}
	\intertext{$K=1$}
	&
	\overbrace
	{
		\left(
			1
			+ \frac{x}{1!}
			+ \frac{x^2}{2!}
			+ \frac{x^3}{3!}
		\right)
	}^
	{
		A
	}
	\overbrace
	{
		\left(
			1
			+ \frac{x}{1!}
			+ \frac{x^2}{2!}
		\right)^2
	}^
	{
		D,I
	}
	\overbrace
	{
		\left(
			1
			+ \frac{x}{1!}
		\right)^5
	}^
	{
		N,I,E,R,V
	}
	\left[
		\frac{x^1}{1!}
	\right] = 8
	\intertext{$K=2$}
	&
	\underbrace
	{
		2!
	}_
	{
		\substack
		{
			\text{El orden}\\
			\text{importa}
		}
	}
	\Bigg{(}
		\overbrace
		{
			\binom{8}{2}
			\frac{1}{1!}
			*\frac{1}{1!}
			* \left(
				\frac{1}{0!}
			\right)^6
		}^
		{
			\substack
			{
				\text{Letras en una palabra}\\
				\text{de largo $2$ con $2$}\\
				\text{letras diferentes}
			}
		}
		+
		\overbrace
		{
			\binom{3}{1}
			* \frac{1}{2!}
			* \left(
				\frac{1}{0!}
			\right)^7
		}^
		{
			\substack
			{
				\text{Letras de palabras de}\\
				\text{largo $2$ con 2 letras}\\
				\text{repetidas}\\
			}
		}
	\Bigg{)}
	\frac{x^2}{2!}\\
	&
	2
	\left(
		\frac{8*7}{2}
		+3*\frac{1}{2}
	\right)=
	56+3
	= 59\
	\intertext{$K=3$}
	&3!
	\Bigg{(}
		\overbrace
		{
			\binom{8}{3}
			* \left(
				\frac{1}{1!}
			\right)^3
			* \left(
				\frac{1}{0!}
			\right)^3
		}^
		{
			\substack
			{
				\text{Todas las letras son iguales}
			}
		}
		+
		\overbrace
		{
			\binom{3}{1}
			\binom{5}{1}
			* \frac{1}{2!}
			* \frac{1}{1!}
			* \left(
				\frac{1}{0!}
			\right)^6
		}^
		{
			\substack
			{
				\text{Hay $2$ letras iguales}\\
				\text{y una distinta solo}\\
				\text{hay una disponible}
			}
		}\\
		&+
		\overbrace
		{
			\underbrace
			{
				\binom{3}{2}
			}_
			{
				\substack
				{
					\text{Cuales $2$}\\
					\text{participan}\\
					\text{de A, D o I}
				}
			}
			\underbrace
			{
				\binom{2}{1}
			}_
			{
				\substack
				{
					\text{Cual de}\\
					\text{ellos está}\\
					\text{duplicada}
				}
			}
			\frac{1}{2!}
			* \frac{1}{1!}
			* \left(
				\frac{1}{0!}
			\right)^6
		}^
		{
			\substack
			{
				\text{Hay $2$ letras iguales}\\
				\text{y una distinta es una}\\
				\text{con más de una disponible}
			}
		}
		+
		\overbrace
		{
			\binom{1}{1}
			* \frac{1}{3!}
			* \left(
				\frac{1}{0!}
			\right)^7
		}^
		{
			\substack
			{
				\text{Todas iguales}
			}
		}
	\Bigg{)}
	\frac{x^3}{3!} + \cdots\\
	& 8*7*6+3*15+3*6+1 = 400
\end{align*}

\end{document}
%}}}
