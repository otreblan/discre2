\makeatletter
\def\input@path{{../}}
\makeatother
\documentclass[../main.tex]{subfiles}

\graphicspath{{ima/clase26}{../ima/clase26}}

% Aquí empieza el documento{{{
\begin{document}
\chapter{Cosas con generatrices exponenciales y grafos}%

\thispagestyle{fancy}

\section{Algo con banderas}%
\label{sec:algo_con_banderas}

Un poste de $n$ metros con Banderas de $1$ metro de alto y de colores
Rojo, Verde, Azul y Blanco.

¿De cuántas formas podemos ordenarlas?

\begin{itemize}
	\item
		\[4^n\]
	\item
		\begin{align*}
			\begin{cases}
				B_n &= 4B_{n-1} \quad n \geq 1\\
				B_1 &= 4
			\end{cases}
			\leadsto
			B^H_n &= C*4^n\\
			B_1 &= 4 = C*4 \Longleftarrow C=1\\
			B_n &= 4^n
		\end{align*}
\end{itemize}

\begin{align*}
	\left(
		\frac{1}{0!}
		+ \frac{x^1}{1!}
		+ \frac{x^2}{2!}
		+ \frac{x^3}{3!}
		+ \cdots
	\right)
	\left(
		\frac{1}{0!}
		+ \frac{x^1}{1!}
		+ \frac{x^2}{2!}
		+ \frac{x^3}{3!}
		+ \cdots
	\right)\\
	\left(
		\frac{1}{0!}
		+ \frac{x^1}{1!}
		+ \frac{x^2}{2!}
		+ \frac{x^3}{3!}
		+ \cdots
	\right)
	\left(
		\frac{1}{0!}
		+ \frac{x^1}{1!}
		+ \frac{x^2}{2!}
		+ \frac{x^3}{3!}
		+ \cdots
	\right)
	\left[
		\frac{x^n}{n!}
	\right]
	\intertext{Multiplicando todo}
	\Bigg{(}
	\Bigg{(}
		\binom{4}{4}
		\left(
			\frac{1}{0!}
		\right)^4
		x^{0+0+0+0}
		+
		\binom{4}{1}
		\frac{1}{1!}
		\binom{3}{3}
		\left(
			\frac{1}{0!}
		\right)^3
		x^{1+0+0+0}
		+\\
		\left(
			\binom{4}{1}
			\frac{1}{2!}
			\binom{3}{3}
			\left(
				\frac{1}{0!}
			\right)^3
			x^{2+0+0+0}
			+
			\binom{4}{2}
			\left(
				\frac{1}{1!}
			\right)^2
			\binom{2}{2}
			\left(
				\frac{1}{0!}
			\right)^2
			x^{1+1+0+0}
		\right)
	\Bigg{)}
	+ \cdots
	\Bigg{)}
	\left[
		\frac{x^n}{n!}
	\right]
	=\\
	\left(
		e^x e^x e^x e^x
	\right)
	\left[
		\frac{x^n}{n!}
	\right]\\
	= e^{4x}
	\left[
		\frac{x^n}{n!}
	\right]=\\
	\left(
		\sum_{k=0}^\infty
		4^k \frac{4^k}{k!}
	\right)
	\left[
		\frac{x^n}{n!}
	\right] = 4^n
\end{align*}

\textbf{Algo de propiedades}
\begin{align*}
	e^x &= \sum_{k=0}^\infty \frac{x^k}{k!} \quad x\in\mathbb{R}\\
	e^{cx} &=
	\sum_{k=0}^\infty \frac{(cx)^k}{k!}\\
	&= \sum_{k=0}^\infty c^k \frac{x^k}{k!}
\end{align*}

\section{Ejercicio exponencial}%
\label{sec:ejercicio_exponencial}

Escriban una identidad interesante del tipo $\sum_{j=0}^\infty\binom{n}{j}=2^n$
usando las ideas de \ref{sec:algo_con_banderas}

\section{Algo con banderas 2}%
\label{sec:algo_con_banderas_2}

Un poste de $n$ metros debe llevar banderas de colores Rojo, Verde, Azul
y Blanco.
Las banderas miden un metro de alto y el poste debe cumplir con:

Una cantidad par de banderas rojas y una cantidad impar de banderas verdes.

\begin{align*}
	&
	\left(
		\frac{1}{0!}
		+ \frac{x^2}{2!}
		+ \frac{x^4}{4!}
		+ \frac{x^6}{6!}
		+ \cdots
	\right)
	\left(
		\frac{x^1}{1!}
		+ \frac{x^3}{3!}
		+ \frac{x^5}{5!}
		+ \cdots
	\right)
	\left(
		\frac{1}{0!}
		+ \frac{x^1}{1!}
		+ \frac{x^2}{2!}
		+ \frac{x^3}{3!}
		+ \cdots
	\right)^2
\end{align*}
\begin{align*}
	\intertext{Sumatoria de los pares}
	\sum_{k=0}^\infty
	\frac{x^{2k}}{(2k)!} &=
	\sum_{j=0}^\infty
	\frac{(x^j+(-x)^j)}{2} * \frac{1}{j!}\\
	&= \frac{1}{2} \sum_{j=0}^\infty
	\frac{x^j}{j!}
	+ \frac{1}{2} \sum_{j=0}^\infty
	\frac{(-x)^j}{j!}\\
	&= \frac{1}{2} e^x
	+ \frac{1}{2} e^{-x}\\
	&= \frac{e^x+e^{-x}}{2} \\
	&= \cosh{x}
	\intertext{Sumatoria de los impares}
	\sum_{k=0}^\infty
	\frac{x^{2k+1}}{(2k+1)!} &=
\end{align*}

\section{Grafos}%
\label{sec:grafos}

Un grafo G es un par ordenado $(V,E)$ con $V$ un conjunto usualmente denominado
``El conjunto de vértices'' y $E$ un subconjunto de
\(
	\underset
	{
		\substack
		{
			\text{Partes de $V$}\\
			\text{tamaño 2.}
		}
	}
	{
		\mathbb{P}_2(V)
	}
\)
denominado ``El conjunto de aristas''


\textbf{Por ejemplo:}

Si $V=\{9,3,\text{:v},*\}$
$\mathbb{P}_2(V)=
\left\{
	\{9,3\}
	,\{9,\text{:v}\}
	,\{9,*\}
	,\{3,\text{:v}\}
	,\{3,*\}
	,\{\text{:v},*\}
\right\} $

Un grafo $G$ prodría ser
\[
\left(
	\overbrace
	{
		\{9,3,\text{:v},*\}
	}^
	{
		V
	}
	,
	\overbrace
	{
		\left\{
			\{9,3\}
			,\{\text{:v},*\}
			,\{*,3\}
		\right\}
	}^
	{
		E \subseteq \mathbb{P}_2(V)
	}
\right)
\]

Otro grafo podría ser:
\begin{align*}
	H &= \left( V, \cancel{o} \right)\\
	K &= \left( V, \mathbb{P}_2(V) \right)
\end{align*}

Observaciones: en el ejemplo
\[
	0 \leq |E| \leq \left| \mathbb{P}_2(V) \right|
\]
Si $V$ es finito $ \left| \mathbb{P}_2(V) \right| = \binom{|V|}{2}$

$G,H,K$ son distintos
\definicion
(Se $G$ un grafo $(V,E)$ $G=(V,E)$ definimos $V(G)=V, E(G)=E$)

Dos grafos $G$ y $H$ son iguales
\[
	\text{si }
	\underset
	{
		\substack
		{
			\text{El conjunto}\\
			\text{de vétices}\\
			\text{de $G$}
		}
	}
	{
		V(G)
	}
	=
	\underset
	{
		\substack
		{
			\text{El conjunto}\\
			\text{de vétices}\\
			\text{de $H$}
		}
	}
	{
		V(H)
	}
	\text{ y }
	E(G) = E(H)
\]

\textbf{Observación:}
\begin{itemize}
	\item La igualdad es muy restrictiva.
\end{itemize}

Una representación gráfica de los grafos mencionados es:

\begin{figure}[H]
	\centering
	\includesvg[width=0.8\linewidth]{dibujo-1}
\end{figure}

Lo que importa es quíen está conectado con quíen.

\begin{figure}[H]
	\centering
	\includesvg[width=0.8\linewidth]{dibujo-2}
\end{figure}

\subsection{Tarea grafos}%
\label{sub:tarea_grafos}

Denme un buen dibujo de
\[
	\mathbb{K}([5])
\]
\end{document}
%}}}
