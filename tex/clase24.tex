\makeatletter
\def\input@path{{../}}
\makeatother
\documentclass[../main.tex]{subfiles}

\graphicspath{{ima/clase24}{../ima/clase24}}

% Aquí empieza el documento{{{
\begin{document}
\chapter{Generadoras exponenciales}%

\thispagestyle{fancy}

\definicion
\textbf{FGO:}

Se usa el coeficiente de $x^n$ para llevar la cuenta
\[
	A(x)[x^n]=
	\underbrace
	{
		a_n
	}_
	{
		\substack
		{
			\text{Coeficiente}\\
			\text{del $x^n$}
		}
	}
	\qquad A(x)=\sum_{k=0}^\infty a_kx^k
\]
Algunas propiedades:
\begin{align*}
	(A+B)(x)[x^n] &= A(x)[x^n]+B(x)[x^n]\\
	(cA)(x)[x^n] &= cA(x)[x^n]\quad c \in \mathbb{R}\\
	(x^kA(x))[x^n] &= A(x)[x^{n-k}]\quad k\in\mathbb{N}^2\\
	(A(x)B(x))[x^n] &=
	\sum_{k=0}^\infty
	a(x)[x^k]B(x)[x^{n-k}]\\
	%
	\left(
		\dv{}{x}A(x)
	\right)
	[x^n] &= (n+1)
	\left(
		A(x)[x^{n+1}]
	\right)\\
	\dv{}{x}A(x) &= \dv{}{x} \sum_{k=0}^\infty a_kn^k\\
	&= \sum_{k=0}^\infty \dv{}{x}a_kn^k\\
	&= \sum_{k=1}^\infty
	ka_kx^{k-1}\\
	&=
	\underbrace
	{
		\sum_{n=0}^\infty
		(n+1)a_{n+1}x^n
	}_
	{
		\substack
		{
			\text{Cambio de variable}\\
			n=k-1
		}
	}
	\\
	%
	\left(
		\int A(x)dx
	\right)
	[x^n] &=
	\begin{cases}
		\frac{1}{n}
		\left(
			A(x)[x^{n-1}]
		\right)
		&n > 0\\
		c &n=0
	\end{cases}\\
	\int A(x)dx &= c+ \int \sum_{k=0}^\infty a_kn^k dx\\
	&= c + \sum_{k=0}^\infty \int a_kx^kdx\\
	&= c + \sum_{k=0}^\infty \frac{a_k}{k+1} x^{k+1}\\
	&= c +
	\underbrace
	{
		\sum_{n=1}^\infty \frac{a_{n-1}}{n} x^n
	}_
	{
		\substack
		{
			\text{Cambio de variable}\\
			n=k+1
		}
	}
\end{align*}

\section{Problemas con generadoras}%
\label{sec:problemas_con_generadoras}

\begin{enumerate}
	\item
		\[
			S_n = \sum_{k=0}^n k e^k\qquad n \geq 0
		\]
		\begin{itemize}
			\item Encuentra $S(x)$. FGO de $\Big{(}S_n\Big{)}_{n=0}^\infty$
			\item Encuentra una fórmula cerrada para $S_n$.
		\end{itemize}
		\begin{align*}
			(1,3,3^2,3^3,\cdots) &\longleftrightarrow
			\sum_{k=0}^\infty 3^kx^k = \frac{1}{1-3x}
			\intertext{Derivada por alguna razón}
			\dv{}{x}
			\left(
				\frac{1}{1-3x}
			\right)
			&= \sum_{k=1}^\infty k3^kx^{k-1}\\
			&= \sum_{n=0}^\infty (n+1)3^{n+1}x^n
			\intertext{Una serie}
			(1*3^1,2*3^2,3*3^3,\cdots) &\longleftrightarrow
			\frac{3}{(1-3x)^2}
			\intertext{$x$ movido}
			(0*3^0,1*3^1,2*3^2,3*3^3,\cdots) &\longleftrightarrow
			\frac{3x}{(1-3x)^2}
			=
			\dv{}{x}
			\left(
				\frac{3x}{(1-3x)^2}
			\right)
			\intertext{Serie de series}
			(0*3^0,0+1*3^1,0+1*3^1+2*3^2,\cdots)
			&\longleftrightarrow
			\underbrace
			{
				\frac{1}{1-x}
			}_
			{
				\substack
				{
					\text{Esta parte}\\
					\text{crea una nueva}\\
					\text{serie usando la}\\
					\text{serie anterior}\\
					\text{como coeficientes}\\
				}
			}
			*\frac{3x}{(1-3x)^2} \\
			\Aboxed
			{
			S(x) &= \frac{3x}{(1-x)(1-3x)^2}
		}
		\end{align*}
		Fracciones parciales
		\begin{align*}
			\frac{3x}{(1-x)(1-3x)^2} &=
			\frac{A}{1-x}
			+ \frac{B}{1-3x}
			+ \frac{3C}{3(1-3x)^2}\\
			3x &=
			A(1-3x)^2
			+B(1-x)(1-3x)
			+C(1-x)\\
			0x^0+3x^1+0x^2 &=
			A(1-6x+9x^2)
			+B(1-4x+3x^2)
			+C(1-x)
			\intertext{Sistema de ecuaciones}
			&\begin{cases}
				0 = A+B+C &x^0 \\
				3 = -6A+4B-C &x^1 \\
				0 = 9A+3B &x^2
			\end{cases}\\
			A &= \frac{3}{4} \\
			B &= -\frac{9}{4} \\
			C &= \frac{6}{4}\\
			\\
			S_n &=
			A \left(
				\frac{1}{1-x}
			\right)[x^n]
			+B \left(
				\frac{1}{1-3x}
			\right)[x^n]
			+ \frac{C}{3}  \left(
				\frac{3}{(1-3x)^2}
			\right)[x^n]\\
			\Aboxed
			{
			S_n &= A+B*3^n+C(n-1)*3^{n-1}
			}
		\end{align*}
\end{enumerate}

\section{Alias generadores}%
\label{sec:alias_generadores}

\begin{gather*}
	\frac{1}{1-x} = \sum_{k=0}^\infty x^k
\end{gather*}
\begin{align*}
	\$ &: \text{ Las sucesiones}\\
	\mathcal{F} &: \text{ Las funciones generadoras}\\
	fgo&: \$ \longrightarrow \mathcal{F}
	\quad fgo\text{ es }|:|\\
	fgo((a_n)_{n=0}^\infty) &=
	\sum_{n=0}^\infty a_n x^n
\end{align*}

\section{Funciones generadoras exponenciales}%
\label{sec:funciones_generadoras_exponenciales}

\textbf{
	La función generadora ordinaria,
	para conteo no toma en cuenta el orden
}
Si queremos tomar en cuenta el orden necesitamos funciones
generadoras exponenciales.
\begin{align*}
	fge &: \$ \longrightarrow \mathcal{F}\\
	fge((a_n)_{n=0}^\infty) &=
	\sum_{k=0}^\infty a_k \frac{x^k}{k!}
\end{align*}

\section{Generadoras para la casa}%
\label{sec:generadoras_para_la_casa}

De cuantas formas podemos ordenar las letras de:


{
	\centering%
	\Huge
	$P=$ANTI
	\(
		\underbrace
		{
			\text{DERI}
		}
	\)
	\(
		\overbrace
		{
			\text{VADA}
		}
	\)

}
Encuentra  una función generadora que resuelva el problema para
cualquier sub-palabra de $P$.

\end{document}
%}}}
