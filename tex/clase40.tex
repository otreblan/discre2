\makeatletter
\def\input@path{{../}}
\makeatother
\documentclass[../main.tex]{subfiles}

\graphicspath{{ima/clase40}{../ima/clase40}}

% Aquí empieza el documento{{{
\begin{document}
\chapter{Árboles}%

\thispagestyle{fancy}

\section*{Tarea:}%
Dado $G$ un grafo planar suficientemente grande demostrar que $G$ tiene al
menos un vértice de grado 5 o menos.
\begin{align*}
	\text{grande}(G) &\Rightarrow \delta(G) \leq 5
\end{align*}

\textbf{Demostración por inducción:}

Si $G$ es planar $\chi(G) \leq 6$.

Pueden usar que si $G$ es planar entonces $|C|+|V|=|E|+2$.

Si $G$ es una triangularización del plano entonces todas las caras son
triángulos y $3|C|=2|E|$

\definicion
Un grafo es un árbol si es conexo y \dobledef{acíclico}{no tiene ciclos}.

\ejemplo

\begin{figure}[H]
	\centering
	\includesvg[width=0.8\linewidth]{dibujo}
\end{figure}

\begin{figure}[H]
	\boldmath
	\bfseries
	\centering
	\includesvg[width=\linewidth]{dibujo-2}
\end{figure}
Al parecer se puede armar un árbol con $n$ vértices añadiendo un vértice y
una arista a un árbol con $(n-1)$ vértices.

Códigos de Prüfer.
Conteo de árboles etiquetados.

\conjetura
Si $T$ es un árbol entonces $|E(T)|=|V(T)|-1$

Por inducción sobre $|E|$ caso base:

\begin{figure}[H]
	\centering
	\includesvg[width=0.8\linewidth]{dibujo-3}
\end{figure}

Hipótesis inductiva:

Si $T$ es un árbol con menos de $n$ vértices entonces $|E(T)|= |V(T)|-1$

Sea $G$ un árbol con $n$ vértices queremos ver que $|E(G)|=|V(G)|-1=n-1$

\begin{figure}[H]
	\centering
	\includesvg[width=0.8\linewidth]{dibujo-4}
\end{figure}

\begin{figure}[H]
	\centering
	\includesvg[width=0.8\linewidth]{dibujo-5}
\end{figure}

\begin{align*}
	V(T_x) &=
	\{
		z\in V(G):
		zPx\text{ no pasa por $y$}
	\}
\end{align*}

Por hipótesis inductiva $E(T_x)=|V(T_x)|-1$ pues $|V(T_x)| < n$
(Y también con $T_y$)

Finalmente
\begin{align*}
	|V(G)| &= |V(T_x)| + |V(T_y)|\\
	|E(G)| &= |E(T_x)| + |E(T_y)| +1\\
	|E(G)| &= |V(T_x)| - 1 + |V(T_y)| - 1 + 1\\
	|E(G)| &= |V(G)| - 1
\end{align*}

\dobledef{Si $G$ es un árbol}{Suficientemente grande}:
\begin{itemize}
	\item \dobledef{$G$ es maximalmente acíclico.}
		{Si le añadimos una arista cualquiera formámos algún ciclo.}
	\item \dobledef{$G$ es minimalmente conexo.}
		{Si le quitamos una arista cualquiera deja de ser conexo.}
	\item Planar.
	\item \dobledef{Bipartito.}{Pues no tiene ciclos. En consecuecia no tiene
		de ciclos de longitud impar.}
\end{itemize}

Si $G$ es planar y conexo entonces:
\[
	|C| + |V| = |E| +
	\underset
	{
		\substack
		{
			\text{Característica}\\
			\text{de Euler.}
		}
	}
	{
		2
	}
\]

\begin{itemize}
	\item Si $G$ no tiene ciclos:
		\[
			|C| = 1 \quad
			\left(
				\substack
				{
					\text{Es un árbol}\\
					\text{}\\
					|E| = |V|-1
				}
			\right)
		\]

	\item Si $G$ tiene ciclos, entonces:
		\[
			|C| > 1
		\]
		Y existen aristas que son frontera entre 2 caras distintas.
		Y están sobre un ciclo.
		Si quitamos la arista frontera, pierdo una cara y una arista sin
		desconectar el grafo.
\end{itemize}

\begin{figure}[H]
	\centering
	\includesvg[width=0.5\linewidth]{dibujo-6}
\end{figure}


\tarea
Si $T$ es un árbol con más de 1 vértice entonces $\delta(T)=1$

Si $\delta(G)\geq 2$ entones $G$ tiene al menos un ciclo.

SI $G$ es un árbol entonces hay un único camino entre $x$ y $y$ para todo
$x,y \in G$.

\end{document}
%}}}
