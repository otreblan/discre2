\documentclass[../main.tex]{subfiles}

\graphicspath{{ima/clase08}{ima}}

% Aquí empieza el documento{{{
\begin{document}

\chapter*{Clase 8}%

\thispagestyle{fancy}

Una función total \underline{booleana} con $n$ parámetros:
\[f:\mathbb{B}^n\longrightarrow\mathbb{B}\]
\[\mathbb{B}=\{0,1\}\]

Podemos pensar que $f$ es un operador $n$-ario (o con aridad $n$)

Si $n=2$ tenemos un operador \underline{binario}, $n>1$ \underline{unario}

\teorema
Toda función booleana se puede describir mediante las operaciones $+$ (join), $\cdot$ (meet) y $\overline{x}$ (complemento)

\definicion
$f:\mathbb{B}^n\longrightarrow\mathbb{B}$ es una \dobledef{conjunción}{disyunción} fundamental si es \dobledef{un producto}{una suma}
de $n$ literales correspondientes a los $n$ parámetros de $f$

Un literal es un parámetro o su complemento
\begin{align*}
	\intertext{Es una conjunción fundamental:}
	&f:\mathbb{B}^4\longrightarrow\mathbb{B}\\
	&f(x_1,x_2,x_3,x_4) = x_1\overline{x_2}x_3x_4
	\intertext{No es una conjunción fundamental:}
	&g:\mathbb{B}^4\longrightarrow\mathbb{B}\\
	&g(x_1,x_2,x_3,x_4) = x_1\overline{x_2}
\end{align*}

\observacion
\dobledef{Conjunción}{Disyunción} fundamental

Es una etiqueta que se asocia a la regla de asignación de la función y no a la función misma.

\noindent No es una conjunción fundamental:
\begin{align*}
	&h:\mathbb{B}^3\longrightarrow\mathbb{B}\\
	&h:(x_1,x_2,x_3)=x_1x_2x_3+(x_1x_2x_3)(x_1\overline{x_2}x_3)
\end{align*}
A pesar de:
\begin{align*}
	x_1x_2x_3+(x_1x_2x_3)(x_1\overline{x_2}x_3) &= x_1x_2x_3+x_1\overline{x_2}x_2x_3\\
	x_1x_2x_3+(x_1x_2x_3)(x_1\overline{x_2}x_3) &= x_1x_2x_3+0\\
	x_1x_2x_3+(x_1x_2x_3)(x_1\overline{x_2}x_3) &= x_1x_2x_3
\end{align*}
Es una disyunción fundamental:
\begin{align*}
	&g:\mathbb{B}^5\longrightarrow\mathbb{B}\\
	&g(x,y,z,w,t) = x+y+z+\overline{t}+\overline{w}
\end{align*}

\definicion
Una función \underline{está} en forma normal \dobledef{disyuntiva}{conjuntiva} si su regla de asignación
es \dobledef{una suma}{un producto} de \dobledef{conjunciones}{disyunciones} fundamentales.
\textcolor{red}{(Creo que están hechos de mintérminos)}
\begin{align*}
	\intertext{Forma normal disyuntiva (FND):}
	&f:\mathbb{B}^3\longrightarrow\mathbb{B}\\
	&f(x_1,x_2,x_3)=\overbrace{x_1x_2x_3}^{\substack{\text{Conjunción}\\ fundamental}}
	+\overbrace{x_1x_2\overline{x_3}}^{\substack{\text{Conjunción}\\ fundamental}}
	+\overbrace{\overline{x_1}x_2\overline{x_3}}^{\substack{\text{Conjunción}\\ fundamental}}
\end{align*}
\begin{align*}
	\intertext{Forma normal conjuntiva(FNC):}
	&g:\mathbb{B}^3\longrightarrow\mathbb{B}\\
	&g(x,y,z)=\overbrace{(x+y+\overline{z})}^{\substack{\text{Disyunción}\\ fundamental}}
	+\overbrace{(\overline{x}+\overline{y}+z)}^{\substack{\text{Disyunción}\\ fundamental}}
\end{align*}
\subsection*{Funciones de un parámetro:}%
\begin{center}
	\begin{tabular}{*{4}{c|}c}
		$x$ & $0$ & $x$ & $\overline{x}$ & $1$\\
		\hline
		0 & 0 & 0 & 1 & 1\\
		1 & 0 & 1 & 0 & 1\\
	\end{tabular}
\end{center}
\subsection*{Funciones de 2 parámetros:}%

\end{document}
%}}}
