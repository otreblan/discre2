\makeatletter
\def\input@path{{../}}
\makeatother
\documentclass[../main.tex]{subfiles}

\graphicspath{{ima/clase33}{../ima/clase33}}

% Aquí empieza el documento{{{
\begin{document}
\chapter{Paseos}%

\thispagestyle{fancy}

\begin{figure}[H]
	\centering
	\includesvg[width=0.6\linewidth]{dibujo}
\end{figure}

\[
	\text{Si } G=(V,E)\qquad \overline{G}=(V_1\mathbb{P}_2 \diagdown E)
\]
\begin{align*}
	G &= (V_G,E_G)\\
	H &= (V_H,E_H)\\
	F &= G \cup H = (V_G \cup V_H, E_G \cup E_H)\\
	G \sqcup H &\leftarrow \text{ Unión disjunta}
\end{align*}

Dado $G=(V,E)$ un grafo, diremos que $H$ es un subgrafo de $G$.
\[
	H \subseteq G
\]
Siempre que $V(H)\subseteq V(G) \wedge E(H) \subseteq E(G)$

\textbf{OBSERVACIÓN:}
De manera disimulada vamos a incluir la idea de isomorfismo dentro de la idea
de subgrafo.

\begin{figure}[H]
	\centering
	\includesvg[width=0.6\linewidth]{dibujo-2}
\end{figure}

En sentido esctricto:
\[
	H \subseteq G \quad \text{pues} \quad 1 \notin V(G)
\]

Pero en general diremos
\[
	H
	\underset
	{\text{\textasciitilde}}
	{
		\subset
	}
	G
\]
Lo que realmente estamos diciendo es hay un $H'\cong H$ con $H'\subseteq G$

Por ejemplo:

Las ruedas con más de 3 vértices tienen triángulos como subgrafos.

\textbf{OBSERVACIÓN:}
\[
	G \subseteq G \forall G
\]
\[
	\phi_0 \subseteq \phi_{|V(G)|}
	\underset
	{
		{\text{\textasciitilde}}
	}
	{
		\subset
	}
	G
\]

\begin{figure}[H]
	\centering
	\includesvg[width=0.6\linewidth]{dibujo-3}
\end{figure}

$n\in \mathbb{B}^2$
\[
	\mathbb{Q}_n =
	\left(
		\{0\}
		\cup
		[2^n-1],
		\left\{
			\{x,y\}:
			\underset
			{
				\substack
				{
					\text{Difieren}\\
					\text{solo en}\\
					\text{un bit}
				}
			}
			{
				x\text{ y }y
			}
		\right\}
	\right)
\]
$\mathbb{Q}_n$ es el $n$-cubo

\begin{figure}[H]
	\centering
	\includesvg[width=\linewidth]{dibujo-4}
\end{figure}

\section{Cosas del grafo cubo}%
\label{sec:cosas_del_grafo_cubo}

\begin{align*}
	|V(\mathbb{Q}_n)|    &= 2^n\\
	|E(\mathbb{Q}_n)|    &= n 2^{n-1}\\
	\delta(\mathbb{Q}_n) &= n\\
	\Delta(\mathbb{Q}_n) &= n
	\intertext{$\mathbb{Q}_n$ es $n$-regular.}
	\sum_{x \in V(\mathbb{Q}_n)} d(x)
	= \sum_{x \in V(\mathbb{Q}_n)} n
	&= n
	\overbrace
	{
		\left(
			\sum_{x\in V(\mathbb{Q}_n)} 1
		\right)
	}^
	{
		|V(\mathbb{Q}_n)|
	}
	= n2^n
	= 2|E(\mathbb{Q}_n)|
\end{align*}

\definicion
Dado $G=(V,E)$ diremos que $H$ es un subgrafo inducido por $\omega \subseteq V$
en $G$ y denotaremos $H=G[\omega]$
\[
	\text{Si } H=(\omega,
	\{
		\{x,y\}\in E(G):
		x\in\omega \wedge y \in \omega
	\}
	)
\]
$H$ es un subgrafo de $G$ con todas sus aristas con ambos extremos en $\omega$.

\begin{figure}[H]
	\centering
	\includesvg[width=\linewidth]{dibujo-5}
\end{figure}

Hay estrellas como subgrafos inducidos de la rueda $\omega_n$

\begin{figure}[H]
	\centering
	\includesvg[width=\linewidth]{dibujo-6}
\end{figure}

\definicion

\dobledef{Un paseo}{en $G$} de largo $n$ es una secuencia alternate de
vértices y aristas de $G$ de maneras que comenzamos y terminamos en un vértice
y las aristas del paseo conectan con los extremos respectivos.

\begin{figure}[H]
	\centering
	\includesvg[width=0.6\linewidth]{dibujo-7}
\end{figure}

\[
	\omega =
	(
		\underset
		{
			\substack
			{
				\text{Inicio}\\
				\text{del}\\
				\text{paseo}
			}
		}
		{
			2
		}
		,
		\overset
		{
			\substack
			{
				\text{El largo del paseo es el}\\
				\text{número de aristas que usa.}
			}
		}
		{
			e_3
			,
			4
			,
			e_5
			,
			5
			,
			\underset
			{
				\substack
				{
					\text{Los extremos}\\
					\text{de $e_5$ son}\\
					\text{4 y 5}
				}
			}
			{
				e_5
			}
			,
			4
			,
			e_4
		}
		,
		\overset
		{
			\substack
			{
				\substack
				{
					\text{Fin}\\
					\text{del}\\
					\text{paseo.}
				}
			}
		}
		{
			3
		}
	)
\]
En un paseo podemos repetir arístas y vértices.

Si el inicio$=$fin el paseo es cerrado, en caso contrario es abierto.

\definicion

Un paseo que no repite aristas se conoce como un sendero.

\definicion
Un sendero que no repite vértices salvo posiblemente el primero y el último se
conoce como camino.

\definicion
Un sendero cerrado es un circuito.

\definicion
Un camino cerrado es un ciclo.

\definicion
$G=(V,E)$ es conexo si entre todo par de vértices de $G$ existe un paseo que
comienza en uno y termina en el otro.

\section{Tarea de paseos}%
\label{sec:tarea_de_paseos}

Demostrar que si hay un paseo de $G$ entre $x$ y $y$ entonces hay un camino
entre $x$ y $y$.

Si miramos el subgrafo correspondiente al camino.

Podemos definir la distancia entre $x$ y $y$ como la consitud del camino más
corto entre $x$ y $y$.

\definicion
Un sendero que visita todas las aristas del grafo se llama de Euler.

\definicion
Un camino que visita todos el vértices del grafo es de Hamilton.

\end{document}
%}}}
