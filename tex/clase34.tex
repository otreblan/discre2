\makeatletter
\def\input@path{{../}}
\makeatother
\documentclass[../main.tex]{subfiles}

\graphicspath{{ima/clase34}{../ima/clase34}}

% Aquí empieza el documento{{{
\begin{document}
\chapter{Paseos y más definiciones}%

\thispagestyle{fancy}

\section{Algo sobre la clase pasada}%
\label{sec:algo_sobre_la_clase_pasada}

Un paseo es una secuencia de vértices y aristas que comienza y termina en un
vértice tal que:

\begin{align*}
	\omega &=
	(
		x_1,e_1,
		x_2,e_2,
		x_3,e_3,
		\cdots,
		x_n,e_n,x_{n-1}
	)\\
	e_i &= \{x_i,x_{i+1}\}\in E(G)\quad \forall i\in[n]
\end{align*}

Notar que esta definición se puede ajustar a versiones diferentes de grafos,
cambiando lo que sea necesario \emph{(Mutatis mutandis)}.

Por ejemplo:

Un camino dirigido es:

\begin{align*}
	(
		x_1,e_1,
		x_2,e_2,
		x_3,\cdots,
		x_n,e_n,x_{n+1}
	)\\
	e_i = (x_i, x_{i+1})\in E(G)
\end{align*}


\section{Paseos}%
\label{sec:paseos}

El largo del paseo es el número de aristas en el.

\begin{align*}
	\omega &=
	(
		x_1,e_1,
		x_2,e_2,
		x_3,\cdots,
		x_n,e_x,x_{n+1}
	)\\
	\text{largo}(\omega) &= n
\end{align*}

\begin{itemize}
	\item Si
	\(
		\overbrace
		{
			x_1
		}^
		{
			\text{Inicio}
		}
		=
		\overbrace
		{
			x_{n+1}
		}^
		{
			\text{Fin}
		}
	\)
	diremos que el paseo es cerrado.

	\item Un paseo que no repite aristas es un sendero.
	\item Un sendero cerrado es un circuito.
	\item Un sendero que no repite vértices
		(salvo posiblemente el primero) es un camino.
	\item Un camino cerrado es un ciclo.
	\item Un \dobledef{camino}{ciclo} que visita todos los vértices de $G$ es
		un \dobledef{Camino}{Ciclo} Hamiltoniano.
		\dobledef{Exactamente una vez.}{Salvo el primero.}
	\item Un \dobledef{sendero}{circuito} que visita todas las aristas de $G$
		es un \dobledef{Sendero}{Circuito} de Euler.
		Exactamente una vez.
\end{itemize}

\observacion
\begin{itemize}
	\item ¿Cómo es un Camino de Euler?
	\item ¿Cómo es un grafo con un Camino de Euler?
	\item ¿Cuándo es un Sendero de Euler un Camino Hamiltoniano?
	\item ¿Cómo es un grafo que tiene un Camino de Hamilton que también es un
		Sendero de Euler?
	\item ¿Cómo es un grafo que tiene un Camino de Hamilton y un Sendero de
		Euler?
\end{itemize}

\begin{figure}[H]
	\centering
	\includesvg[width=0.4\linewidth]{dibujo}
\end{figure}

\begin{align*}
	\omega_1 &=
	(
		\overbrace
		{
			1,
			\underbrace
			{
				2,
				3,
				4,
				2,
			}_
			{
				\substack
				{
					\text{Vértices}\\
					\text{internos}
				}
			}
			5
		}^
		{
			\text{Extremos}
		}
	)
	&& \substack
	{
		\text{Un paseo. Un Sendero.}\\
		\text{No es un camino pues}\\
		\text{repite ``2'' un vértice interno.}
	}
	\\
	\omega_2 &= (1,2,3,4)
	&& \substack
	{
		\text{ Un \dobledef{camino}{abierto}.}
	}
	\\
	\omega_3 &= (1,2,3,4,2,5,1)
	&& \substack
	{
		\text{ Un circuito de Euler.}
	}
\end{align*}

\section{Tarea de paseos 2}%
\label{sec:tarea_de_paseos_2}

\begin{enumerate}
	\item Muestra por inducción que \dobledef{$Q_n$}{el $n$-cubo}
		siempre tiene al menos un Camino Hamiltoniano.
	\item Recuerda que si $G=(V,E)$ es tal que
		$d(x) \in \dot{2} \quad \forall x \in V$

		Entonces $G$ tiene un Circuito de Euler.
		\subitem Es decir
		\(
			\underset
			{
				n\in\mathbb{N}
			}
			{
				Q_{2n}
			}
		\)
		tiene Caminos Hamiltonianos y Circuitos de Euler.
\end{enumerate}


\begin{figure}[H]
	\centering
	\includesvg[width=0.6\linewidth]{dibujo-2}
\end{figure}

\definicion
Un grafo $G=(V,E)$ es conexo si $\forall x,y \in V$ existe un paseo entre $x$ y
$y$.

Usualmente un paseo de esa forma de denota.
\[
	x\mathbb{W}y
\]

\[
	x\mathbb{W} y =
	\underbrace
	{
		x\mathbb{W}_1y\mathbb{W}_2z
	}_
	{
		\mathbb{W}
	}
\]

\begin{figure}[H]
	\centering
	\includesvg[width=0.6\linewidth]{dibujo-3}
\end{figure}


\observacion

La relación

$x\mathbb{C}y$: $x$ está conectado a $y$ mediante algún paseo en $G$.

Es:
\begin{itemize}
	\item Reflexiva.
	\item Simétrica.
	\item Transitiva.
\end{itemize}

$\mathbb{C}$ define una relación de equivalencia sobre los vértices de $G$.

\definicion

$G$ es conexo si $\forall x,y\in V \Rightarrow x\mathbb{C}y$

Es decir, todos los vértices de $G$ están en la misma clase de equivalencia.

\definicion
Dado $G=(V,E)$ $nc(G)$ es el número de clases de equivalencia de conectitud
en $G$.

\begin{figure}[H]
	\centering
	\includesvg[width=0.6\linewidth]{dibujo-4}
\end{figure}

{
	\Huge
	\bfseries
	\centering
	\textcolor{red}
	{
		CUIDADO
	}

}

Si el grafo es dirigido la conectitud es más dificil de ver.

\begin{figure}[H]
	\centering
	\includesvg[width=0.9\linewidth]{dibujo-5}
\end{figure}

\definicion
Si
\(
	\overset
	{
		\text{finito}
	}
	{
		G
	}
\)
es conexo. Entonces:
\begin{align*}
	\text{dist}(x,y) &= \min{|E(x\mathbb{P}y)|}\\
	\text{dist}_{G_1}(1,4) &= 3\\
	\text{dist}_{G_2}(1,8) &= \infty\\
	\text{dist}_{G_2}(1,6) &= 2= \text{dist}_{H}(6,1)
\end{align*}

Se sobrecarga la idea de distancia para incluir $\infty$ como distancia entre
vértices en distintas componentes conexas.

\section{Tarea parte 3}%
\label{sec:tarea_parte_3}

Si $\text{dist}_{G}(x,y)=k\qquad k\in\mathbb{N}^*=\{0\}\cup\mathbb{N}$

\begin{itemize}
	\item ¿Qué podemos decir sobre $\text{dist}_H(x,y)$ si $H\subseteq G$?
		$x,y\in V(H)$
	\item ¿Qué podemos decir sobre $\text{dist}_F(x,y)$ si $F=G[w]$?
		Para algún $w$ con $x\in w, y\in w$.
		$x,y\in V(F)\quad w\subseteq V(G)$
\end{itemize}

Calcula $nc(G)$ para los grafos que ya definimos
\[
	\mathbb{K}_n, \omega_n, \mathbb{S}_n, \mathbb{A}_n, \cdots
\]

\definicion
$\mathbb{K}_{n,m}$: el grado bipartito completo

\begin{align*}
	V(\mathbb{K}_{n,m})=
	\{
		(i,j):
		&\text{ con }i\in [2],\\
		&j \in[n]\text{ si } i=1,\\
		&j\in[m]\text{ si }i=2
	\}
\end{align*}
\begin{align*}
	E(\mathbb{K}_{n,m})=\{\{i,j\},\{k,l\}:i\neq k\}
\end{align*}

$\mathbb{K}_{1,n}$:
\begin{align*}
	V(\mathbb{K}_{i,n}) &=
	\{
		(1,1),
		(2,1),
		(2,2),
		\cdots,
		(2,n)
	\}\\
	E(\mathbb{K}_{i,n}) &=
	\{
		\{(1,1),(2,1)\},
		\{(1,1),(2,2)\},
		\{(1,1),(2,3)\},
		\cdots,
		\{(1,1),(2,n)\}
	\}
\end{align*}

\[
	\mathbb{K}_{1,n} \cong \mathbb{S}_n
\]

\begin{figure}[H]
	\centering
	\includesvg[width=0.4\linewidth]{dibujo-6}
\end{figure}

\[
	\mathbb{K}_{1,n}
	\cong
	\mathbb{K}_{n,1}
\]

$\mathbb{K}_{2,3}$

\[
	V(\mathbb{K}_{2,3})=
	\{
		(1,1),
		(1,2),
		(2,1),
		(2,2),
		(2,3),
	\}
\]

\begin{figure}[H]
	\centering
	\includesvg[width=0.6\linewidth]{dibujo-7}
\end{figure}

\subsection*{Demuestra}%
\begin{align*}
	\overline{\mathbb{K}}_{n,m} &\cong \mathbb{K}_n \sqcup \mathbb{K}_m\\
	\overline{\mathbb{K}}_{n,m} \cup \mathbb{K}_{n,m} &\cong \mathbb{K}_{n+m}
\end{align*}

\end{document}
%}}}
