\documentclass[../main.tex]{subfiles}

\graphicspath{{ima/clase15}{ima}}

% Aquí empieza el documento{{{
\begin{document}
\chapter{Recurrencias}%

\thispagestyle{fancy}

\definicion

Una \underline{Ecuación de recurrencia} tiene una suceción como solución y
ocasionalmente puede venir de problemas de conteo.
\[
	\begin{cases}
		P_n=nP_{n-1}\quad n \geq 1 \longleftarrow
		\substack
		{
			\text{Factorial}\\
			\text{o permutaciones}\\
			\text{de $n$ objetos}
		}\\
		P_0=1
	\end{cases}
\]

\[
	\begin{cases}
		\binom{n+1}{k}
		=\binom{n}{k}
		+\binom{n}{k-1}\quad
		\substack
		{
			0\leq k \leq n+1\\
			0\leq k-1 \leq n\\
			0\leq k \leq n\\
		}
		\\
		%
		\binom{0}{0}
		=1\\
		\binom{n}{0}
		=\binom{n}{n}
		=1
	\end{cases}
\]
\section{Torres de hanoi}%
\label{sec:Torres de hanoi}
\begin{itemize}
	\item Hay que pasar todos los discos del poste $A$ al poste $C$.
	\item Los discos solo pueden moverse:
		\begin{itemize}
			\item Uno a la vez
			\item A un palo adyacente.
			\item Encima de un disco de mayor diámetro.
		\end{itemize}
\end{itemize}

\begin{figure}[H]
\begin{center}
\begin{tikzpicture}[scale=1, transform shape]
	\foreach \x in {0, ..., 2}
	{
		\draw (\x*3,0) rectangle +(0.5,5);
		\draw (\x*3,-0.5) node {\AlphAlph{\x+1}};
	}
	\foreach \y in {0, ..., 3}
	{
		\filldraw [fill=white](0,\y)+(-2+\y/3,0) rectangle +(2-\y/3,1);
	}
\end{tikzpicture}
\end{center}
\end{figure}
Pienses en un ``Algoritmo'' para mover todos los discos.($n$ discos)

¿Cuál es la menor cantidad de pasos en que se puede reolver el problema?

\teorema
$T_n$: \# de movimiento necesarios para resolver torres de hanoi con $n$ discos.

\begin{center}
	\begin{tabular}{c|c}
		$n$ & $T_n$\\
		\hline
		$1$ & $2$ \\
		$2$ & $8$ \\
		$3$ & $26$ \\
		$4$ & $80$
	\end{tabular}
\end{center}

\[
	\begin{cases}
		T_{n+1}=3T_n+2\\
		T_1=2
	\end{cases}
\]

\begin{gather*}
	\sum^\mathbb{N}_{i=1}i
	= \frac{n(n+1)}{2} \\
	S_n =
	\sum^n_{i=1}i\\
	\begin{cases}
		S_{n+1} = S_n+(n+1)\\
		S_1 = 1
	\end{cases}
\end{gather*}

\begin{gather*}
	Q_n
	\sum^n_{k=0} r^k\quad r\in\mathbb{R}\quad r\neq 0\\
	\begin{cases}
		Q_{n+1}=Q_n+r^{n+1}\\
		Q_{n+1}=rQ_n+1
	\end{cases}\\
	r^0+rQ_n= r^0+r
	\sum^n_{k=0} r^i\\
	r^0+rQ_n= r^0+
	\sum^n_{k=0} r^{i+1}\\
	r^0+rQ_n= r^0+
	\underbrace
	{1}_{j=i+1}+
	\sum^{n+1}_{j=1}r^j=Q_{n+1}\\
	\\
	\begin{cases}
		Q_{n+1} = Q_n+
		\underbrace
		{
			r^{n+1}
		}_{
			\substack
			{
				\text{Parte no}\\
				\text{homogenea}\\
				\text{de la recurrencia}
			}
		}
		\quad n\geq1\quad r\neq 0\quad r\neq 1\\
		Q_0=1
	\end{cases}\\
	\begin{cases}
		\underbracket
		{
			Q_{n+1} = rQ_n
		}_
		{
			\substack
			{
				\text{Grado de la}\\
				\text{recurrencia, esta}\\
				\text{es de orden $1$}
			}
		}
		+
		\underbrace
		{
			1
		}_
		{
			\substack
			{
				\text{Parte no}\\
				\text{homogenea}\\
				\text{de la recurrencia}
			}
		}
		\quad n \geq 1\quad r \neq 0\\
		\underbrace
		{
			Q_0=1
		}_
		{
			\substack
			{
				\text{Una condición}\\
				\text{inicial}
			}
		}
	\end{cases}\\
	\intertext{Igualando ambas versiones de la serie}
	Q_n+r^{n+1}=rQ_n+1\\
	(1-r)Q_n = 1-r^{n+1}\\
	Q_n =
	\frac{1-r^{n+1}}{1-r}
\end{gather*}
Ecuación característica:
\begin{gather*}
	x^{n+1}=rx^n\\
	x=r
\end{gather*}
\section{Fibonacci}%
\label{sec:Fibonacci}

\begin{gather*}
	\begin{cases}
		\overbrace
		{
			\underbracket
			{
				f_n = f_{n-1}+f_{n-2}
			}_
			{
				\substack
				{
					\text{Recurrencia de}\\
					\text{orden $2$}
				}
			}
		}^
		{
			\text{Recurrencia homogenea}
		}
		\quad \text{La parte no homogenea es $0$}
		\\
		f_0 = 0\quad\text{Dos condiciones}\\
		f_1 = 1\quad\text{iniciales}
	\end{cases}
\end{gather*}

\begin{center}
	\begin{tabular}{c|c}
		$n$ & $f_n$\\
		\hline
		$0$ & $0$ \\
		$1$ & $1$ \\
		$2$ & $1$ \\
		$3$ & $2$ \\
		$4$ & $3$ \\
		$5$ & $5$ \\
		$6$ & $8$
	\end{tabular}
\end{center}
Esta es una recurrencia \underline{lineal}

Ecuación característica:
\[
	x^n= x^{n-1}+x^{n-2}\rightarrow\boxed{x^2=x+1}
\]

\begin{gather*}
	\begin{cases}
		P_n =
		\underbrace
		{
			n
		}_
		{
			\substack
			{
				\text{No es}\\
				\text{\bfseries LINEAL}
			}
		}
		P_{n-1}\quad n \geq 1\\
		P_0 = 1
	\end{cases}
\end{gather*}
Es recurrencia homogenea de orden $1$

\begin{gather*}
	\begin{cases}
		T_{n+3} = T_{n-5} +
		\underbrace
		{
			2
		}_
		{
			\substack
			{
				\text{Recurrencia}\\
				\text{no-homogenea}\\
				\text{lineal de orden 8}
			}
		}
		-3T_n\quad n \geq 6\\
		T_i = i\quad 1 \leq i \leq 8
	\end{cases}
\end{gather*}
Ecuación característica:
\begin{gather*}
	x^{n+3} = x^{n-5}-3x^n\\
	x^8 = 1-3x^5
\end{gather*}
\definicion
Si tenemos un $\overbrace{\text{E.R}}^{\substack{\text{Ecuación}\\\text{en recurrencia}} }$,
lineal, de orden $K$ podemos definir la \underline{ecuación característica}
de la E.R reemplazando $a_n$ por $x^n$ y simplificando el polinomio.
\textcolor{red}{\bfseries Debemos ignorar la parte no homogenea}

¿Qué pasa si $r$ satisface la ecuación característica?
(Supongamos que $r_1$ y $r_2$ son raíces de $x^2=x+1$)
\begin{gather*}
	r^2=r+1
	\intertext{Multiplico por $r^{n-2}$}
	Ar^n = Ar^{n-1}+ Ar^{n-2}
\end{gather*}
Propongo $f_n = Ar^n$
\[
	Ar^n?=Ar^{n-1}+Ar^{n-2}
\]

\begin{align*}
	x^2 &= x+1\\
	x^2-x-1 &= 0\\
	\\
	r_1 &= x^+=
	\frac{1+\sqrt{1+4}}{2}=
	\frac{1+\sqrt{5}}{2}\\
	r_2 &= x^-=
	\frac{1-\sqrt{1+4}}{2}=
	\frac{1-\sqrt{5}}{2}\\
\end{align*}

\subsection{Solución general de fibonacci:}%
\label{sub:fiboGeneral}

\begin{gather*}
	\begin{cases}
		f_n = Ar_1^n+Br_2^n\\
		f_0 = 0 = A+B\\
		f_1 = 1 = Ar_1+Br_2
	\end{cases}
	A=-B\\
	1=-Br_1+Br_2\\
	1=B(r_2-r_1)\\
	B= \frac{1}{r_2-r_1}\\
	\\
	f_n =
	\frac{\sqrt{5}}{5}
	\left(
		\frac{1+\sqrt{5}}{2}
	\right)^n
	-
	\frac{\sqrt{5}}{5}
	\left(
		\frac{1-\sqrt{5}}{2}
	\right)^n
\end{gather*}

\end{document}
%}}}
