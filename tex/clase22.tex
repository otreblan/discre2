\makeatletter
\def\input@path{{../}}
\makeatother
\documentclass[../main.tex]{subfiles}

\graphicspath{{ima/clase22}{../ima/clase22}}

% Aquí empieza el documento{{{
\begin{document}
\chapter{Herramientas generedoras}%

\thispagestyle{fancy}

Si $A(x),B(x)$ son \dobledef{$FGO$}{Funciones generedora ordinaria}
de $a_n,b_n$ y $c\in\mathbb{R}$ entonces:
\begin{align*}
	\sum_{n=0}^\infty
	(a_n+b_n)
	x^n
	&=
	A(x)+B(x) = (A+B)(x)\\
	%
	\sum_{n=0}^\infty
	ca_nx^n &= cA(x)\\
	\\
	(A+B)(x)[x^n] &= A(x)[x^n]+B(x)[x^n]=a_n+b_n\\
	(cA)(x)[x^n] &= cA(x)[x^n] = ca_n
\end{align*}

\begin{align*}
	\frac{d}{dx} A(x)
	&= \sum_{n=1}^\infty
	\frac{d}{dx} a_nx^n
	= \sum_{n=1}^\infty
	na_nx^{n-1}
	= \sum_{k=0}^\infty
	(k+1)a_{k+1}x^k\\
	%
	SA(x)dx &=
	c+
	\sum_{n=0}^\infty
	\int anx^ndx
	= c+
	\sum_{n=0}^\infty
	\frac{a_nx^{n+1}}{n+1}
	\overset
	{
		n+1=j
	}
	{
		=
	}
	c+
	\sum_{j=1}^\infty
	\frac{a_{j-1}}{j} x^j
\end{align*}

\begin{align*}
	\left(
		\frac{d}{dx} A(x)
	\right)
	[x^n] &= (n+1)a_{n+1}\\
	\left(
		\int A(x)dx
	\right)
	[x^n] &=
	\begin{cases}
		c &n=0\\
		\frac{a_{n-1}}{n} &n > 0
	\end{cases}
\end{align*}

$m\in\mathbb{Z} \quad m \leq 0$
\begin{align*}
	x^m A(x) &= x^m
	\sum_{j=0}^\infty
	a_jx^j\\
	&=
	\sum_{j=0}^\infty
	a_jx^{j+m}\\
	&
	\underset
	{
		k=j+m
	}
	{
		=
	}
	\sum_{k=m}^\infty
	a_{k-m}x^k
\end{align*}

\begin{align*}
	(x^m A(x))[x^n] &=
	\begin{cases}
		0 &n < m\\
		a_{n-m} &n \geq m
	\end{cases}
\end{align*}

Más cosas para la caja de herramientas:
\begin{align*}
	(a_0,a_1,a_a,\cdots) &\longleftrightarrow A(x)\\
	(
	\underbrace
	{
		0,0,\cdots,0
	}_
	{
		\text{$m$ ceros}
	}
	,a_0,a_1,a_2,\cdots
	) &\longleftrightarrow x^mA(x)\\
	(1*a_1,2*a_2,3*a_3,4*a_4,\cdots) &\longleftrightarrow \frac{d}{dx} A(x)\\
	(
		c,
		\frac{a_0}{1},
		\frac{a_1}{2},
		\frac{a_2}{3},
		\frac{a_3}{4},
		\cdots
	)
	&\longleftrightarrow
	\int A(x)dx
\end{align*}

\section{Algo con distributiva infinita}%
\label{sec:algo_con_distributiva_infinita}
\[
	(A(x)B(x))[x^n]
\]
\begin{align*}
	A(x)B(x) &=
	\left(
		\sum_{n=0}^\infty
		a_nx^n
	\right)
	\left(
		\sum_{k=0}^\infty
		b_kx^k
	\right)
	= (a_0+a_1x+a_2x^2+\cdots)
	(b_0+b_1x+b_2x^2+\cdots)
	\\
	&= (a_0b_0)x^0
	+(a_1b_0+a_0b_1)x^1
	+(a_2b_0+a_1b_1+a_0b_2)x^2 + \cdots
\end{align*}
En la posición $n$: ($a$ baja y $b$ sube)
\begin{align*}
	(a_nb_0+a_{n-1}b_1+a_{n-2}b_2+\cdots+a_1b_{n-1}+a_0b_n)x^n
\end{align*}
\[
	\left((A\cdot B)(x)\right)[x^n] =
	\sum_{j=0}^n
	a_{n-j}b_j
\]

Cosas extrañas:
\[
	\left(
		\frac{1}{1-x} \cdot A(x)
	\right)
	[x^n] =
	\sum_{j=0}^na_j
\]
\[
	(
		a_0,
		a_0+a_1,
		a_1+a_2,
		a_2+a_3,
		\cdots
	)
	\longleftrightarrow
	\frac{1}{1-x} A(x)
\]
\[
	\frac{1}{1-x} =
	\sum_{k=0}^\infty
	x^k
	\longleftrightarrow
	(1,1,1,1,1,\cdots)
\]
El $S(x)$ es como una sumatoria de sumatorias.

Su $n$ elemento es una sumatoria.
\[
	S_n = \sum_{j=0}^n j
\]
\begin{align*}
	(0,1,2,3,4,5,\cdots)
	&\longleftrightarrow
	F(x)
	= \frac{x}{(1-x)^2}
	\\
	(
		0,
		0+1,
		0+1+2,
		0+1+2+3,
		\cdots
	)
	&\longleftrightarrow
	\frac{1}{1-x} F(x)
	= \frac{1}{1-x}
	\cdot \frac{x}{(1-x)^2}
	= S(x)
\end{align*}

\begin{align*}
	(1,2,3,4,\cdots)
	&\longleftrightarrow
	H(x)= \frac{1}{(1-x)^2}\\
	(0,1,2,3,\cdots)
	&\longleftrightarrow
	xH(x)= \frac{x}{(1-x)^2}\\
	\\
	(1,1,1,1,\cdots)
	&\longleftrightarrow
	\frac{1}{1-x} \\
	(1,1+1,1+1+1,1+1+1+1,\cdots)
	&\longleftrightarrow
	\frac{1}{1-x}
	\cdot
	\frac{1}{1-x}
	= \frac{1}{(1-x)^2}
\end{align*}
Fracciones parciales:
\[
	\underbrace
	{
		S(x) = \frac{x}{(1-x)^3} =
		x\left(
			\frac{A}{1-x}
			+\frac{B}{(1-x)^2}
			+\frac{C}{(1-x)^3}
		\right)
	}_
	{
		\text{Esto no funciona}
	}
\]
Intentando resolverlo:
\[
	(1,1,1,\cdots)
	\longleftrightarrow \frac{1}{1-x}
\]
\[
	\frac{d}{dx}
	\sum_{k=0}^\infty x^k
	=
	\sum_{k=1}^\infty kx^{k-1}
	\underset{k-1=j}{=}
	\sum_{j=0}^\infty (j+1)x^j
\]
\begin{align*}
	(1,2,3,4,\cdots)
	&\longleftrightarrow
	\frac{1}{(1-x)^2}
	= \frac{d}{dx}
	\left(
		\frac{1}{1-x}
	\right)\\
	&=
	\frac{d}{dx}
	\left(
		\frac{1}{(1-x)^2}
	\right)= \frac{-2*1*(-1)}{(1-x)^3}\\
	&= \frac{2}{(1-x)^3}\\
	(2*1,3*2,4*3,5*4,\cdots)
	&\longleftrightarrow
	\sum_{j+1}^\infty jx^{j-1}\\
	&\longleftrightarrow
	\sum_{k=0}^\infty (k+2)(k+1)x^k
\end{align*}
\begin{align*}
	\frac{2}{(1-x)^3}[x^n] &= (n+2)(n+1)\\
	\\
	\left(
		\frac{x}{(1-x)^3}
	\right)
	[x^n]
	&= \frac{1}{(1-x)^3}
	[x^{n-1}]\\
	&=
	\left(
		\frac{1}{2}
		\cdot
		\frac{2}{(1-x)^3}
	\right)
	[x^{n-1}]\\
	&= \frac{1}{2}
	\left(
		\frac{2}{(1-x)^3}
	\right)
	[x^{n-1}]\\
	&= \frac{1}{2}
	(n-1+2)(n-1+1)\\
	&= \frac{1}{2} (n+1)(n)
\end{align*}
\section{Suma de cuadrados(Nada de esto funciona)}%
\label{sec:suma_de_cuadrados}
\begin{align*}
	C_n &= \sum_{j=1}^n j^2\\
		&= \sum_{j=1}^n j(j-1+1)\\
		&= \sum_{j=1}^n (j(j-1)+j)\\
		&= \sum_{j=1}^n j(j-1) +
		\boxed
		{
			\sum_{j=1}^n j
		}
\end{align*}

\begin{align*}
	C_n =
	\sum_{j=1}^n j(j-1)
	+ \sum_{j=1}^n
	\overset
	{
		\text{FGO}
	}
	{
		\longleftrightarrow
	}
	&\frac{1}{1-x}
	\cdot \frac{2x}{(1-x)^3}
	+ \frac{1}{(1-x)^3}\\
	&= \frac{2x+(1-x)}{(1-x)^4} \\
	&= \frac{x+1}{(1-x)^4} \\
	&= \frac{x}{(1-x)^4} + \frac{1}{(1-x)^4}
\end{align*}
Derivando:(No sé de donde sale esta derivada)
\begin{align*}
	\frac{d}{dx}
	\left(
		\frac{2}{(1-x)^3}
	\right)
	&=
	2(-3)(1-x)^{-4}(-1)\\
	&= \frac{6}{(1-x)^4} \\
	&= \frac{d}{dx}
	\left(
		\sum_{k=0}^\infty (k+2)(k+1)x^k
	\right)\\
	&= \sum_{k=1}^\infty (k+2)(k+1)kx^{k-1}
	\intertext{Reindexando}
	&= \sum_{n=0}^\infty (n+3)(n+2)(n+1)x^n\\
	\left(
		\frac{6}{(1-x)^4}
	\right)
	[x^n]
	&=
	(n+3)(n+2)(n+1)
\end{align*}
Algo extraño
\begin{align*}
	C_n &=
	\left(
		\frac{x}{(1-x)^4}
		+ \frac{1}{(1-x)^4}
	\right)
	[x^n]\\
	&= \frac{1}{6}
	\left(
		\frac{6}{(1-x)^4}
	\right)[x^{n-1}]
	- \frac{1}{6}
	\left(
		\frac{6}{(1-x)^4}
	\right)[x^n]\\
	&= \frac{1}{6}(n+2)(n+1)n
	+ \frac{1}{6}(n+3)(n+2)(n+1)\\
	&= \frac{(n+2)(n+1)(2n+3)}{6}
\end{align*}
Esto sería la respuesta correcta:
\[
	\sum_{k=1}^n k^2 = \frac{n(n+1)(2n+1)}{6}
\]

\end{document}
%}}}
