\makeatletter
\def\input@path{{../}}
\makeatother
\documentclass[../main.tex]{subfiles}

\graphicspath{{ima/clase38}{../ima/clase38}}

% Aquí empieza el documento{{{
\begin{document}
\chapter{Cosas extrañas}%

\thispagestyle{fancy}

\definicion

El número de clique de un grafo es el $n$ tal que $\mathbb{K}_n$ es subgrafo
de $g$ y $n$ es lo más grande posible.
\[
	\text{clique}(G) = \max n
\]

\definicion

Un grafo $g$ es planar si tiene una representación plana.

\definicion

Un a representación plana de un grafo es plana si es un dibujo del grafo donde
las aristas son curvas en el plano que solo tocan vértices en sus extremos.

\begin{figure}[H]
	\boldmath
	\centering
	\includesvg[width=0.6\linewidth]{dibujo}
\end{figure}

\begin{align*}
	g &\cong \mathbb{K}_4\\
	h &\cong \mathbb{K}_4
\end{align*}

La representación de $g$ es plana.

La representación de $h$ es plana.

$\mathbb{K}_4$ es planar.

\subsection*{Ejemplo}%
$\mathbb{K}_5$ no es planar.

$\mathbb{K}_{3,3}$ no es planar.

Demostrar que un grafo ``general'' no es planar es ``dificil''.

\teorema
Kuratowsky: $G$ es planar si no tiene a $\mathbb{K}_5$ o $\mathbb{K}_{3,3}$
como menor.

\end{document}
%}}}
