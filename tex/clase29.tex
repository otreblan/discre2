\makeatletter
\def\input@path{{../}}
\makeatother
\documentclass[../main.tex]{subfiles}

\graphicspath{{ima/clase29}{../ima/clase29}}

% Aquí empieza el documento{{{
\begin{document}
\chapter{Grafos sin levantar el lápiz}%

\thispagestyle{fancy}

\begin{align*}
	G&=(V,E)\text{ un grafo. \dobledef{Finito}{Podemos suponer que $V$ es $[n]$}}\\
	A&=ADY(G)\\
	\\
	%A_{ij} &=
	%\begin{cases}
		%aaa
	%\end{cases}
	A_{ij}&\in\{0,1\} \forall i\neq j\\
	A_{ii}&= 0\quad\forall i \in V
\end{align*}

Algo sobre cambio de nombre a los nodos.

\begin{figure}[H]
	\centering
	\includesvg[width=0.8\linewidth]{dibujo}
\end{figure}

\begin{figure}[H]
	\centering
	\includesvg[width=0.8\linewidth]{dibujo-2}
\end{figure}

\begin{align*}
	A_{DY}(G_i)=
	a\quad
	b\quad
	c\quad
	d\quad
	e\quad
	f\quad
	g\quad\\
	\begin{bmatrix}
		0 & 1 & 0 & 0 & 0 & 0 & 0\\
		1 & 0 & 0 & 0 & 0 & 0 & 0\\
		0 & 0 & 0 & 0 & 0 & 0 & 0\\
		0 & 0 & 0 & 0 & 1 & 1 & 1\\
		0 & 0 & 0 & 1 & 0 & 0 & 0\\
		0 & 0 & 0 & 1 & 0 & 0 & 1\\
		0 & 0 & 0 & 1 & 0 & 1 & 0
	\end{bmatrix}\\
	A_{DY}(H_i)=
	1\quad
	2\quad
	3\quad
	4\quad
	5\quad
	6\quad
	7\quad\\
	\begin{bmatrix}
		0 & 0 & 0 & 0 & 1 & 0 & 0\\
		0 & 0 & 0 & 0 & 0 & 0 & 0\\
		0 & 0 & 0 & 1 & 0 & 1 & 1\\
		0 & 0 & 1 & 0 & 0 & 0 & 1\\
		1 & 0 & 0 & 0 & 0 & 0 & 0\\
		0 & 0 & 1 & 0 & 0 & 0 & 0\\
		0 & 0 & 1 & 1 & 0 & 0 & 0
	\end{bmatrix}
\end{align*}

Observación:
\begin{align*}
	A_{DY}(G_i) &\neq A_{DY}(H_i)\\
	\text{pero } G_I &\cong H_i
\end{align*}

Es decir, existe un reordenamiento $P$ de filas y columnas de $A_{DY}(G_i)$
llamemosla $B$ que hacen que $A_{DY}(G_i)=B$

\begin{figure}[H]
	\centering
	\includesvg[width=0.8\linewidth]{dibujo-3}
\end{figure}

\section{Algo con dominós}%
\label{sec:algo_con_dominos}

En el juego de dominó doble $n$ hay piezas rectangulares.

\begin{figure}[H]
	\centering
	\includesvg[width=0.4\linewidth]{dibujo-4}
\end{figure}

\begin{itemize}
	\item ¿Cuántas piezas tiene el dominó si hay una
		copia de cada pieza posible?
		\begin{align*}
			\frac{(n+1)^2-(n+1)}{2}
			+(n+1)
			&=
			\binom{n+2}{2}
		\end{align*}
	\item Dibuja un grafo que represente el dominó \dobledef{doble $4$.}
		{Dominó doble 4}
\end{itemize}

\begin{figure}[H]
	\centering
	\includesvg[width=0.8\linewidth]{dibujo-5}
\end{figure}

Euler (Recorrer el grafo pasando por todas las aristas solo una vez)

\begin{figure}[H]
	\centering
	\includesvg[width=0.8\linewidth]{dibujo-6}
\end{figure}

¿Podemos hacer esto en un dominó doble-3?.

No porque es grado de todos los vértices es impar.

\begin{figure}[H]
	\centering
	\includesvg[width=0.3\linewidth]{dibujo-7}
\end{figure}

Algo extraño (Los números son el grado de cada vértice)
\begin{figure}[H]
	\centering
	\includesvg[width=0.8\linewidth]{dibujo-8}
\end{figure}

Dado $G$ un grafo y $x$ un vértice en el grafo.
\begin{align*}
	d(x) &=
	\left|
		\underbrace
		{
			\{y\in V\}: \{x,y\}\in E
		}_
		{
			\text{$V_x$: es el vecindario de $x$.}
		}
	\right|
	\text{Es el grado de $x$.}\\
	2|E| &= \sum_{x\in V}d(x)
\end{align*}

\begin{figure}[H]
	\centering
	\includesvg[width=0.8\linewidth]{dibujo-9}
\end{figure}

\dobledef{Un grafo cualquiera}
{Con al menos dos vértices}
tiene a lo mucho un número par de vértices de grado impar.
($\dot{2}$: múltiplo de 2)
\begin{align*}
	2|E| &=
	\sum_{x\in V}d(x)
	=
	\underset
	{
		d(x)\in \dot{2}
	}
	{
		\sum_{x\in V}
	}
	d(x)
	+
	\underset
	{
		d(x)\notin \dot{2}
	}
	{
		\sum_{x\in V}
	}d(x)\\
	\underset
	{
		d(x)\notin \dot{2}
	}
	{
		\sum_{x\in V}
	}
	d(x)
	&=
	2|E|
	-
	\underset
	{
		d(x)\in \dot{2}
	}
	{
		\sum_{x\in V}
	}
	d(x)
	\text{Tiene que ser par}
\end{align*}
Todo grafo de 2 vértices o más tiene al menos 2 vértices del mismo grado.

%TODO
{
	\centering
	\bfseries

	\#TODO

	Un grafo extraño.

	Acepto pull requests.

	\url{https://github.com/otreblan/discre2}

}
\end{document}
%}}}
