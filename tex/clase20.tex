\makeatletter
\def\input@path{{../}}
\makeatother
\documentclass[../main.tex]{subfiles}

\graphicspath{{ima/clase20}{../ima/clase20}}

% Aquí empieza el documento{{{
\begin{document}
\chapter{Coeficientes indeterminados}%

\thispagestyle{fancy}

Dada una recurrencia para $a_n$ (Una sucesión (Una función
de $\{0\}\cup\mathbb{N}\rightarrow\mathbb{R}$)) donde

(Esto es como una plantilla para crear recurrencias)
\[
	\begin{cases}
		a_n =
		\overbrace
		{
			\sum_{j=1}^kc_ja_{n-j}
		}^
		{
			\substack
			{
				\text{Una recurrencia de}\\
				\text{orden $k$ si $c_k\neq 0$}\\
			}
		}
		\underbrace
		{
			+ f(n)
		}_
		{
			\substack
			{
				\text{Si esto es cero}\\
				\text{la recurrencia es homogénea}\\
				\text{en caso contrario}\\
				\text{no es homogénea}
			}
		}
		\quad \forall n \geq k\\
		\begin{rcases}
			a_0 = b_0\\
			a_1 = b_1\\
			\vdots\\
			a_{k-1} = b_{k-1}
		\end{rcases}
		\underbrace
		{
			\text{Condiciones iniciales}
		}_
		{
			\substack
			{
				\text{T. si se tiene una recurrencia lineal de orden $k$ para}\\
				\text{$a_n$ bastan $k$ condiciones iniciales consecutivas para}\\
				\text{determinar de manera única a $a_n$}
			}
		}
	\end{cases}
\]


\begin{align*}
	\text{Con } c_i&\in\mathbb{R} \text{ constantes fijas para } i\in[k]\\
	b_i&\in\mathbb{R} \text{ constantes fijas para } i\in\{0\}\cup[k-1]\\
\end{align*}
y $f(n)$ una combinación lineal de polinomios en $n$ por exponenciales en $n$.
\[
	f(x)=\sum_{j=1}^\omega P_j(n)d_j^n \quad
	d_j\in\mathbb{R} \text{ Fijos para $j\in[\omega]$}
\]

\[
	\begin{cases}
		a_n = 7a_{n-4}+3 \quad n \geq 4\\
		a_0 = 1\\
		a_1 = 2\\
		a_2 = 3\\
		a_3 = 4
	\end{cases}
\]
Método general para resolver recurencias.
\begin{enumerate}
	\item \textcolor{red}
		{
			\bfseries\boldmath%
			Econtrar la ecuación carácterística. ($i=\sqrt{-1}$)
		}
		\begin{align*}
			x^4 &= 7\\
			r_1 &= \sqrt[4]{7}\\
			r_2 &= -\sqrt[4]{7}\\
			r_3 &= \sqrt[4]{7}i\\
			r_4 &= -\sqrt[4]{7}i\\
			x^4 &= 7 \Rightarrow (x^4-7) = 0 \Rightarrow\\
			(x-\sqrt[4]{7})
			(x+\sqrt[4]{7})
			(x-\sqrt[4]{7}i)
			(x+\sqrt[4]{7}i) &= 0
		\end{align*}
	\item \textcolor{red}
		{
			\bfseries\boldmath%
			Proponer la solución homogénea.
			Si las $k$ raíces de la ecuación carácterística son.
			$r_1,r_2,\cdots,r_s$ con multiplicidades
			(Las repeticiones de una raiz)
			$m_1,m_2,\cdots,m_s$ respectivamente.
			$m_i\in\{0\}\cup\mathbb{N}$ y $\sum_{i=1}^smi=k$
			proponemos $a_n^h = \sum_{i=1}^sp_i(n)r_i^n$
			donde $p_i(n)=\sum_{j=0}^{m_i-1}A_{ij}n^j$
			(Es un polinomio de grado $m_i-1$ y $A_{ij}$
			(Los $A.B.C$ que había usado para fibonacci)
			constante)
		}
		\[
			a_n^h =
			A_{1,0}\sqrt[4]{7}^n
			+
			A_{2,0}\left(-\sqrt[4]{7}\right)^n
			+
			A_{3,0}\left(\sqrt[4]{7}i\right)^n
			+
			A_{4,0}\left(-\sqrt[4]{7}i\right)^n
		\]
	\item \textcolor{red}
		{
			\bfseries\boldmath%
			%
			Considerar la parte no-homogénea para encontrar una
			relación particular $a_n^p$
			%
			Hay que pensar que relación carácterística tendríamos que
			tener para ``ver'' la $f(n)$
		}
		\[
			f(n) = 3 =
			\underbrace
			{
				3
			}_
			{
				\substack
				{
					\text{Está repetida}\\
					\text{una vez}
				}
			}
			*
			\underbrace
			{
				1^n
			}_
			{
				\substack
				{
					\text{La raiz}\\
					\text{es $1$}
				}
			}
			\rightarrow (r-1)^1 = 0
		\]
		Como $1$ no es raíz de la parte homogénea entonces $q_n^p=A\cdot1^n$

		Sustituimos en la recurrencia para encontrar los valores de las
		constantes.
		\begin{align*}
			A*1^n &= 7*A*1^{n-4}+3*1^n\\
			A &= 7A+3\\
			-6A &= 3\\
			\Aboxed
			{
			A &= \frac{1}{2}
			}
		\end{align*}
	\item \textcolor{red}
		{
			\bfseries\boldmath%
			%
			Continuamos la solución general.
			$a_n = a_n^h+a_n^p$
		}
		\[
			a_n =
			A_{1,0} \sqrt[4]{7}^n
			A_{2,0} \left(-\sqrt[4]{7}\right)^n
			A_{3,0} \left(\sqrt[4]{7}i\right)^n
			A_{4,0} \left(-\sqrt[4]{7}i\right)^n
			- \frac{1}{2}
		\]
		\begin{align*}
			1 &= A_{1,0}
			+
			A_{2,0}
			+
			A_{3,0}
			+
			A_{4,0}
			+
			\left( \frac{-1}{2} \right)\\
			2 &= A_{1,0} \sqrt[4]{7}
			-
			A_{2,0} \sqrt[4]{7}
			+
			A_{3,0} \sqrt[4]{7}i
			-
			A_{4,0} \sqrt[4]{7}i
			-
			\frac{1}{2}\\
			3 &= A_{1,0} \sqrt[4]{7}^2
			+
			A_{2,0} \sqrt[4]{7}^2
			-
			A_{3,0} \left(\sqrt[4]{7}i\right)^2
			-
			A_{4,0} \left(\sqrt[4]{7}i\right)^2
			-
			\frac{1}{2}\\
			4 &= A_{1,0} \sqrt[4]{7}^3
			-
			A_{2,0} \sqrt[4]{7}^3
			-
			A_{3,0} \sqrt[4]{7}^3i
			-
			A_{4,0} \sqrt[4]{7}^3i
			-
			\frac{1}{2}\\
		\end{align*}
		Resuelve $A_{1,0}, A_{2,0}, A_{3,0}, A_{4,0}$
\end{enumerate}
\section{Ejercicios random}%
\label{sec:ejercicios_random}

Un poste con $n$ metros de altura se quiere llenar con banderas de colores.
Las banderas tienen un metro de alto y pueden ser rojas, verdes o azules.
\begin{enumerate}
	\item\label{it:normal} ¿De cuántas formas podemos hacer esto?
		\[
			3^n
		\]
	\item Plantea una recurrencia
		\[
			\begin{cases}
				f_n = 3f_{n-1} \quad f \geq 1\\
				f_0 = 1\\
				f_1 = 3
			\end{cases}
		\]
	\item Resuelve la recurrencia y comparala con \ref{it:normal}
		\begin{align*}
			\intertext{Ecuación carácterística}
			f^n &= 3f^{n-1}\\
			f^n &= 3 \frac{f^n}{f} \\
			f &= 3\\
			\\
			f_n^h &= A*3^n\\
			\\
			f_1 &= 3 = A*3^1 \Rightarrow A=1
		\end{align*}
\end{enumerate}

Un poste con $n$ metros de altura se quiere llenar con banderas de colores.
Las banderas tienen un metro de alto y pueden ser rojas, verdes o azules.
O de dos metros de alto blancas o negras.
\begin{enumerate}
	\item\label{it:normal2} ¿De cuántas formas podemos hacer esto?
		\[
			?
		\]
	\item Plantea una recurrencia
		\[
			\begin{cases}
				f_n = 3f_{n-1}+ 2f_{n-2} \quad f \geq 3\\
				f_0 = 1\\
				f_1 = 3\\
				f_2 = 11 =
				\underbrace
				{
					3^2
				}_
				{
					\substack
					{
						\text{Patrones con $2$}\\
						\text{banderas de $1m$}
					}
				}
				+
				\overbrace
				{
					2
				}^
				{
					\substack
					{
						\text{Patrones con $1$}\\
						\text{bandera de $2m$}
					}
				}
			\end{cases}
		\]
	\item Resuelve la recurrencia y comparala con \ref{it:normal2}
		\begin{align*}
			\intertext{Ecuación carácterística}
			f^n &= 3f^{n-1}+2f^{n-2}\\
			f^n &= 3 \frac{f^n}{f} +2 \frac{f^n}{f^2}\\
			1 &= \frac{3}{f} + \frac{2}{f^2}\\
			f^2 &= 3f + 2\\
			f^2 - 3f - 2 &= 0\\
			\\
			f^+ = r_1 &= \frac{3+\sqrt{17}}{2} \\
			f^- = r_2 &= \frac{3-\sqrt{17}}{2}\\
			\\
			f_n^h &= A(r_1)^n + B(r_2)^n\\
			f_n^h &=
			A\left(
				\frac{3+\sqrt{17}}{2}
			\right)^n
			+
			f\left(
				\frac{3-\sqrt{17}}{2}
			\right)^n\\
			\\
			f_0 &= 1 = A + B\\
			f_1 &= 3 =
			A
			\left(
			\frac{3+\sqrt{17}}{2}
			\right)^1
			+
			B
			\left(
			\frac{3-\sqrt{17}}{2}
			\right)^1
		\end{align*}
\end{enumerate}
\textbf{Para llevar}
Un poste con $n$ metros de altura se quiere llenar con banderas de colores.
Las banderas tienen un metro de alto y pueden ser rojas, verdes o azules.
\begin{enumerate}
	\item ¿De cuántas formas podemos poner las banderas sin que hayan
		$2$ banderas rojas consecutivas?
	\item ¿De cuántas formas podemos poner las banderas sin que hayan
		$2$ banderas del mismo color consecutivas?
\end{enumerate}

\section{Otro torres de Hanoi}%
\label{sec:otro_torres_de_hanoi}
Los postes están puestos en un triángulo.

\textbf{Los 9 billones de nombres de Dios}
Arthur Clarke

\begin{figure}[H]
	\centering
	\includesvg[width=0.6\linewidth]{dibujo}
\end{figure}

Queremos llevar los discos de $A$ a $C$.
\begin{itemize}
	\item No podemos mover más de $1$ disco por vez.
	\item No puede haber un disco grande encima de uno pequeño.
\end{itemize}

\[
	\begin{cases}
		H_n = H_{n-1}+1+H_{n-1}\\
		H_1 = 1
	\end{cases}
\]
\[
	\begin{cases}
		H_n = 2H_{n-1} + 1\\
		H_1 = 1
	\end{cases}
\]
\begin{align*}
	H^n &= 2H^{n-1}\\
	H &= 2\\
	\intertext{Parte homogénea}
	H_n^h &= A*2^n
	\intertext{Parte no-homogénea(Solución particular)}
	H_n^p &= B*1^n = B\\
	B*1^n &= 2B*1^n +1\\
	B &= 2B+1 \quad B=-1
	\intertext{Combinando ambas partes}
	H_n &= A*2^n-1\\
	\\
	H_1 &= 1 = A*2^1-1\\
	2 &= A*2\\
	A &= 1\\
	\\
	H_n &= 2^n -1
\end{align*}
\textbf{Para llevar}
\[
	\begin{cases}
		S_n = S_{n-1} + n^2\\
		S_0 = 0
	\end{cases}
\]
Con función generadora:
\[
	\begin{cases}
		S_n = S_{n-1} + n \\
		S_0 = 0
	\end{cases}
\]
Todo multiplicado por $x^n$
\begin{gather*}
	\sum_{n=1}^\infty
	S_nx^n =
	\sum_{n=1}^\infty
	\left(
		S_{n-1}x^n+nx^n
	\right)\\
	S(x) - S_0 =
	\sum_{n=1}^\infty
	S_{n-1}x^n
	+
	\sum_{n=1}^\infty
	nx^n\\
	=
	\sum_{n=1}^\infty
	S_{n-1}x^{n-1}
	+
	\sum_{n=1}^\infty
	nx^n\\
	\\
	S(x) - 0 = xS(x)+
	x
	\sum_{n=1}^\infty
	nx^{n-1}<\\
	\\
	\frac{d}{dx}
	\left(
	\sum_{n=0}^\infty
	x^n
	\right)
	=
	\sum_{n=1}^\infty
	\frac{d}{dx}
	x^n =
	\sum_{n=1}^\infty
	nx^{n-1}\\
	\frac{d}{dx}
	\left(
		\frac{1}{1-x}
	\right)=
	(-1)(1-x)^{-2}(-1)=
	\frac{1}{(1-x)^2} \\
	\frac{d}{dx}
	(1-x)^{-1}
\end{gather*}
\begin{gather*}
	S(x) = xS(x) + \frac{x}{(1-x)^2} \\
	(1-x)S(x) = \frac{x}{(1-x)^2}\\
	S(x) = \frac{x}{(1-x)^3}
\end{gather*}
Estamos buscando el coeficiente de $x^n$ en $S(x)$
\[
	\left(
		S(x)
	\right)
	\left[
		x^n
	\right]
	=
	\frac{x}{(1-x)^3}
	\left[
		x^n
	\right]
\]
\[
	\left(
		\sum_{n=2}^\infty
		\frac{n(n-1)}{2}
		x^{n-1}
	\right) =
	S(x) \quad
	S(x)[x^k] = \frac{(k+1)(k+1-1)}{2} =
	\boxed
	{
		\frac{(k+1)k}{2}
	}
\]
Derivada otra vez
\begin{gather*}
	\frac{d}{dx}
	\left(
		\frac{1}{(1-x)^2}
	\right)
	=
	(-2)(1-x)^{-3}(-1) = \frac{2}{(1-x)^3} \\
	\frac{d}{dx}
	\left(
		\sum_{n=1}^\infty
		nx^{n-1}
	\right)
	\underbrace
	{
		=
	}_
	{
		\substack
		{
			nx^{1-1}\\
			\text{es constante}\\
			\text{respecto a $x$.}
		}
	}
	\sum_{n=2}^\infty
	n(n-1)
	x^{n-2}
\end{gather*}
\begin{gather*}
	\sum_{n=2}^\infty
	\frac{n(n-1)}{2}
	x^{n-2} = \frac{1}{(1-x)^3} \\
	x\sum_{n=2}^\infty
	\frac{n(n-1)}{2}
	x^{n-2} = \frac{x}{(1-x)^3} = S(x)
\end{gather*}
\end{document}
%}}}
