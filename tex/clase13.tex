\documentclass[../main.tex]{subfiles}

\graphicspath{{ima/clase13}{ima}}

% Aquí empieza el documento{{{
\begin{document}
\chapter*{Clase 13}%

\thispagestyle{fancy}

$|A|<\infty$\quad
$|B|<\infty$
\begin{align*}
	f&:A\underset{1:1}{\longrightarrow}B\\
	|A| &= |B|
\end{align*}

Una ficha de dominó doble-n consiste en un ``Rectángulo'' con un número entre
0 y n en cada extremo.

\textcolor{red}{El dominó doble-n tiene todas las fichas posibles.}

¿Cuántas fichas tiene el dominó doble-6?
¿Doble-9?
\[
	(\underbrace{6*6}_{\substack{\text{Todos los}\\\text{pares}\\\text{ordenados} }}-
	\underset{\text{Los dobles}}{6}
	)\underset{\substack{\text{El orden}\\\text{no importa}} }{/2!}
	+\underset{\text{dobles}}{6}=
	\binom{6}{2}
	+\binom{6}{1}
\]

\[
	\underbrace
	{ \begin{cases}
			x_1+x_2+x_3=10 \\
			x_i \geq 1 \quad x_1 \in \mathbb{Z}
	\end{cases} }_
	{\text{\btw}}
	\longleftrightarrow
	\underset{\binom{7+3-1}{3-1}}
	{
		\begin{cases}
			x_1'+x_2'+x_3'=7\\
			x_1 \geq 0 \quad x_i \in \mathbb{Z}
		\end{cases}
	}
\]
\begin{align*}
	|A| &= \text{Soluciones de \btw}\\
	|A| &= \binom{7+3-1}{3-1} = \binom{9}{2}=36\\
	\frac{\binom{9}{2}-4*3}{3!} +4 &= 8
\end{align*}
Un mazo de baraja francesa consta de un número de cartas que se forman tomando
un rango del conjunto $\{A,2,3,4,5,6,7,8,9,10,J,Q,K\}$
y una pinta del conjunto $\{\heartsuit,\spadesuit,\diamondsuit,\clubsuit\}$

¿Cuántas cartas tiene un mazo de baraja francesa?
\[
	(13)*(4)=52
\]

Una mano de pocker consta de 5 cartas seleccionadas al azar de un mazo de baraja francesa bien mezclado.
¿Cuántas manos de poker son posibles?
\[
	\binom{52}{2}
\]

Un \underline{Full house} es una mano de poker con $3$ cartas de un rango y $2$
cartas de otro rango.
¿Cuántos full-houses son posibles?
\[
	\Big{(}
	\overbrace{13}^{\text{Números}}
	*
	\underbrace{ \binom{4}{3} }_{\substack{\text{$3$ cartas de $4$}\\ \text{pintas del mismo}\\
	\text{número}} }
	\Big{)}
	*
	\Big{(}
	\overbrace{12}^{\substack{\text{Números}\\\text{diferentes}} }
	*
	\underbrace{ \binom{4}{2} }_{\substack{\text{$2$ cartas de $4$}\\ \text{pintas del mismo}} }
	\Big{)}
\]

Un doble par consiste de:

$2$ pares de cartas del mismo rango pero distinto entre sí y una carta de un rango
distint de estos.
\[
	2^\heartsuit2^\diamondsuit3^\clubsuit3^\diamondsuit K^\clubsuit
\]
¿Cuántos dobles pares hay?
\[
	\overbrace{ \binom{13}{2} }^{\substack{\text{Dos números}\\\text{del rango}} }
	\underbrace{ \binom{4}{2}^2 }_{\text{Dos pares de 2}}
	\overbrace{ \binom{11}{1} }^{\substack{\text{Un número}\\\text{del rango}\\\text{restante}} }
	\underbrace{ \binom{4}{1} }_{\text{Una carta}}
\]
Integrantes:\\
Renato Rodríguez Llanos\\
Frings Barrueta Aspajo\\
Jarod Rojas\\
Alberto Oporto\\
Ricardo Torres\\
\end{document}
%}}}
