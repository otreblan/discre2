\documentclass[../main.tex]{subfiles}

\graphicspath{{ima/clase16}{ima}}

% Aquí empieza el documento{{{
\begin{document}
\chapter{Más recurrencia}%

\thispagestyle{fancy}

\definicion
\[
	\begin{cases}
		a_1 =
		\left(
			\sum_{i=1}^k
			c_ia_{n-i}
		\right)
		+
		\overbrace
		{
			f_n
		}^
		{
			\substack
			{
				\text{Es la parte}\\
				\text{no homogénea.}\\
				\text{no puede tener a's}
			}
		}
		\quad\quad n\geq k\\
		\begin{rcases*}
			a_0 = b_0\\
			a_1 = b_1\quad\quad b_i \in\mathbb{R}\\
			a_2 = b_2\quad\quad c_i \in\mathbb{R}\\
			\vdots\\
			a_{n-1} = b_{n-1}
		\end{rcases*}
		\substack
		{
			\text{Caso base}\\
			\text{condiciones}\\
			\text{iniciales}
		}
	\end{cases}
\]
Una ecuación en recurrencia para
\(
	\left(
		a_n
	\right)_{n=0}^\infty
\)

\textcolor{red}{\bfseries Lineal}
\textcolor{blue}{\bfseries\boldmath De orden $K\quad (c_k\neq0)$}

\section{Receta}%
\label{sec:Receta}

\begin{itemize}
	\item \dobledef{Identificar la ecuación característica y resolverla}
		{Encontrar todas la raíces}
\end{itemize}
La solución a la parte homogénea es de la forma
\[
	\sum_{j=1}^{S}
	\left(
		\sum_{i=1}^{m_j}
		A_{ij}
		n^{i-1}
		r_j
	\right)
\]
Donde $m_j$ es el número de veces que se repite la raíz $r_j$ en la ecuación
característica y $S$ es el número de raices diferente que tiene la relación.

Si la recurrencia es homogénea solo falta utilizar las condiciones iniciales
para determinar los valores de las $A_{ij}$

SI la recurrencia es no-homogénea hay que identificar ``la forma''
que tiene para proponer una solución particular.
\ref{sec:Recurrencia no-homogénea}

{
	\Large\bfseries
	\textcolor{green}{Por ejemplo:}
}

\textcolor{green}
{
	\bfseries\boldmath
	Si la ecuación caractarística fuese:
	\[
		(x-5)^2
		(x+1)^3
		(x-7)
		(x+2)
		=0
	\]
	La solución para la parte homogénea sería:
	\[
		\overset{A_{11}}
		{
			A5^n
		}
		\overset{A_{12}}
		{
			+B_n5^n
		}
		\overset{A_{21}}
		{
			+C(-1)^n
		}
		\overset{A_{22}}
		{
			+D_n(-1)^n
		}
		\overset{A_{23}}
		{
			+En(-1)^n
		}
		\overset{A_{31}}
		{
			+F7^n
		}
		\overset{A_{41}}
		{
			+G(-2)^n
		}
	\]
}
\[
	\sum_{j=1}^{S}m_j = k
	\substack
	{
		\text{El orden}\\
		\text{de la}\\
		\text{recurrencia}
	}
\]

Por ejemplo:
\[
	\begin{cases}
		d_n = 4d_{n-1}-4d_{n-2}\quad n \geq 2\\
		\\
		d_0=1\\
		d_1=2
	\end{cases}
\]
\begin{center}
	\begin{tabular}{c|c}
		$n$ & $d_n$\\
		\hline
		$0$ & $1$\\
		$1$ & $2$\\
		$2$ & $4$\\
		$3$ & $8$\\
		$4$ & $16$\\
		$\vdots$ & $\vdots$
	\end{tabular}
\end{center}

E.C.
\begin{align*}
	x^n  &=  4x^{n-1}-4x^{n-2}\\
	x^2 &= 4x-4\\
	(x^2-4x+4) &= 0\\
	(x-2)^2 &= 0\\
	x &= 2
\end{align*}

Solución general:\ref{sub:fiboGeneral}
\[
	A2^n+
	\underbrace
	{
		B_n2^n
	}_
	{
		\substack
		{
			\text{Demuestra que}\\
			\text{satisfase}\\
			\text{la recurrencia}
		}
	}
\]

\begin{align*}
	B_n2^n &=
	2^2
	\left(
		B(n-1)2^{n-1}
	\right)
	-2^2
	\left(
		B(n-2)2^{n-2}
	\right)\\
	&= Bn2^{n+1}
	-\cancel{B2^{n+1}}
	-Bn2^n
	+\cancel{B2^{n+1}}\\
	&= Bn2^{n+1}
	-Bn2^n
	=Bn2^n
\end{align*}

\begin{gather*}
	\begin{cases}
		d_0 = 1 = A2^0+B*0*2^0\\
		d_1 = 2 = A2^1+B*1*2^1
	\end{cases}\\
	A=1\\
	2A+2B=2
	\Longrightarrow
	B=0
	\Longrightarrow
	d_n=2^n
\end{gather*}

\section{Recurrencia no-homogénea}%
\label{sec:Recurrencia no-homogénea}


\begin{gather*}
	\begin{cases}
		h_n=4h_{n-1}-4h_{n-2}+
		\overbrace
		{
			n+3^n
		}^
		{
			\substack
			{
				\text{Parte}\\
				\text{no-homogénea}
			}
		}
		\\
		\\
		h_0=1\\
		h_1=11
	\end{cases}
\end{gather*}

\begin{align*}
	f_n &= n + 3^n \\
	f_n &=
	( 0*n^0*\underbrace{1^n}_{(x-1)})
	+(1*n^1*\underbrace{1^n}_{(x-1)})
	+(1*n^0*\underbrace{3^n}_{(x-3)})\\
\end{align*}
\[
	\overbrace
	{
		(x-1)^2(x-3)
	}^
	{
		\text{Ecuación característica}
	}
\]

Propongo
\[
	\boxed
	{
		\overbrace
		{
			A
		}
		^{(x-1)}
		+
		\overbrace
		{
			Bn
		}
		^{(x-1)}
		+
		\overbrace
		{
			C3^n
		}
		^{(x-3)}
	}
	\quad\substack
	{
		\text{Como}\\
		\text{solución}\\
		\text{particular}
	}
\]

\begin{align*}
	A+Bn+C3^n &=
	4(A+B(n-1)+C3^{n-1})\\
	& -4(A+B(n-2)+C3^{n-2})
	+n+3^n\\
	A+Bn+C3^n &=
	\cancel{4A}+\cancel{4Bn}-4B+4C3^{n-1}\\
	\cancel{-4A}+\cancel{-4Bn}-8B+4C3^{n-2}
	+n+3^n\\
	\\
	A&=4B\\
	Bn&=n\Longrightarrow B=1 \Longrightarrow A = 4\\
	C3^n &= 4C3^{n-1}-4C3^{n-2}+3^n\\
	9C&=12C-4C+9\\
	C&=9
\end{align*}

$4+n+9*3^n$ es una solución particular de mi recurrencia no-homogénea

\section{Algo}%
\label{sec:Algo}
\[
	h_n^4 = h_n^H+h_n^P
\]
\[
	h_n=A2^n+B_n2^n+4+n+9*3^n
\]
\begin{align*}
	h_0 &= 1=A*2^0+B*0*2^0+4+0+9*3^0\\
	h_0 &= 11=A*2+B*1*2+4+1+9*3\\
	&\begin{cases}
		1 = A+4+9 \Longrightarrow A=-12\\
		11 = 2A+2B+4+1+27 \Longrightarrow B= \frac{3}{2}
	\end{cases}
\end{align*}

\[
	\begin{cases}
		\omega_n = 4\omega_{n-1}-4\omega_{n-2}+2n^n+1\\
		\\
		\omega_0 = 1\\
		\omega_1 = 5
	\end{cases}
\]
\[
	\omega_n^H=A2^n+Bm2^n
\]
Propuesta para solución particular:
\[
	(x-1)(x-2)^2
\]
\[
	A+Bn^22^n+Cn^32^n
\]

\begin{align*}
	A+Bn^22^n+Cn^32^n&=
	4(A+B(n-1)^22^{n-1}+C(n-1)^32^{n-1})\\
	&-4(A+B(n-2)^22^{n-2}+C(n-2)^32^{n-2})
	+n2^n+1\\
	%
	A+Bn^22^n+Cn^32^n&=
	\cancel{4A}+4Bn^22^{n-1}-8nB2^{n-1}+4B2^{n-1}\\
	&+4C(n^3-3n^2+3n-1)2^{n-1}\\
	&\cancel{-4A}-4B(n^2-4n+4)2^{n-2}-4C(n^3-6n^2+12n-8)2^{n-2}\\
	&+n2^n+1
\end{align*}
\begin{align*}
	A&=1\\
	n^22^n\\
	B&=\cancel{2B-6C-B+6C}\\
	n^32^n\\
	C&=\cancel{2C-C}\\
	n2^n\\
	0&=\cancel{-4B}+6C+\cancel{4B}-12C+1\\
	C &= \frac{1}{6}\\
	2^n\\
	0&=2B-2C-4B+8C\\
	0&=-2B+6C\\
	0&=-2B+1\\
	B &= \frac{1}{2}
\end{align*}

Solución particular:
\[
	\omega_n^P1+ \frac{1}{2}n^22^n+\frac{1}{6}n^32^n
\]
\[
	\omega_n=A2^n+Bn2^n+1+\frac{1}{2}n^22^n+\frac{1}{6}n^32^n
\]

\end{document}
%}}}
